\documentclass[11pt]{article}
\usepackage[margin=1in]{geometry}
\usepackage{hyperref}
\usepackage{amsmath,amssymb}
\usepackage{graphicx}
\usepackage{booktabs}
\usepackage{enumitem}
\title{ Federlicht Report - 20260112\_ms-ai-diffusion }
\author{ Hyun-Jung Kim / AI Governance Team }
\date{ 2026-01-14 }
\begin{document}
\maketitle

\noindent\textit{Federlicht assisted and prompted by "Hyun-Jung Kim / AI Governance Team" — 2026-01-14 21:55}

\section{Abstract}
본 리뷰는 Microsoft AI Economy Institute가 2026년 1월 공개한 보고서 Global AI Adoption in 2025 — A Widening Digital Divide를 단일 1차 근거로 삼아, 2025년 상반기(H1)와 하반기(H2) 사이 전세계 생성형 AI 사용(확산) 변화와 지역, 국가, 플랫폼 요인의 상호작용을 설명형으로 정리한다. 핵심 변화는 두 가지로 요약된다. 첫째, 전세계 생성형 AI 도구 사용 비중(AI diffusion)이 H2 2025에 16.3\%로 상승해(H1 2025 15.1\% 대비 +1.2\%p) “대략 6명 중 1명” 수준에 도달했다는 점이다. 둘째, 같은 기간 Global North와 Global South의 확산 격차가 9.8\%p에서 10.6\%p로 확대되었고, H2 증가폭 상위 10개국이 모두 고소득 국가라는 관찰이 결합되며 ‘성장’과 ‘불균형’이 동시에 가속되는 국면이 확인된다. 보고서는 확산 지표를 집계·익명화된 Microsoft 텔레메트리 기반으로 산출하되 OS/디바이스 점유율, 인터넷 보급률, 인구 차이를 보정하는 인구 정규화 방식으로 정의하며, 단일 지표의 한계와 향후 보완 필요성도 명시한다 \href{./archive/web/text/Microsoft-AI-Diffusion-Report-2025-H2.txt}{[1]}. 본 리뷰는 이 지표 정의와 수치, 국가 사례(대한민국, UAE), 그리고 DeepSeek의 오픈소스·지정학적 확산 서사를 연결해 2025년 ‘확산의 방향성’을 전략적 관점에서 해석한다.

\section{Introduction}
\subsection{검토 범위와 자료 선정}
본 문서는 단일 출처 기반 연간(반기) 변화 리뷰다. 입력은 Microsoft Research의 PDF 보고서 1건이며, 동일 PDF의 텍스트 추출본과 탐색용 Tavily extract가 보조적으로 활용되었다 \href{./archive/20260112\_ms-ai-diffusion-index.md}{[2]}, \href{./instruction/20260112\_ms-ai-diffusion.txt}{[3]}, \href{./archive/web/text/Microsoft-AI-Diffusion-Report-2025-H2.txt}{[1]}, \href{./archive/tavily\_extract/0001\_https\_www.microsoft.com\_en-us\_research\_wp-content\_uploads\_2026\_01\_Microsoft-AI-Diffusion-Report-2025-H2.pdf.txt}{[4]}. 외부 웹 자료는 본 런의 별도 supporting 디렉토리로 수집된 것이 아니므로, 본문에서 1차 실증 근거로 확장 해석하지 않고, 보고서가 인용한 참고문헌은 “보고서 내부의 주장 구조”를 이해하는 맥락으로만 취급한다(예: Edelman Trust Barometer, Bloomberg 등은 보고서가 인용했음을 확인하는 수준) \href{./archive/web/text/Microsoft-AI-Diffusion-Report-2025-H2.txt}{[1]}.

\subsection{시간 창과 비교 기준}
보고서의 비교 축은 2025년 상반기(H1 2025)와 하반기(H2 2025)이며, 지표는 “해당 기간 동안 생성형 AI 제품을 사용한 인구 비중”으로 정의된다 \href{./archive/web/text/Microsoft-AI-Diffusion-Report-2025-H2.txt}{[1]}. 따라서 본 리뷰에서의 “연간 변화”는 엄밀한 12개월 연간치라기보다, 동일 연도 내 반기 간 변화(확산 속도, 격차의 방향)를 추적하는 Annual Review 스타일의 합성으로 이해하는 것이 적절하다.

\subsection{핵심 지표(AI diffusion)의 의미와 해석상 주의}
보고서는 AI diffusion을 Microsoft 텔레메트리에서 도출한 사용 기반 지표로 정의하고, 국가 간 비교 가능성을 높이기 위해 OS/디바이스 점유율, 인터넷 보급률, 인구 차이를 반영해 보정한다고 밝힌다. 동시에 “No single metric is perfect”라며 단일 지표의 불완전성을 인정하고, 추가 지표로 보완하겠다고 명시한다 \href{./archive/web/text/Microsoft-AI-Diffusion-Report-2025-H2.txt}{[1]}. 이 문구는 본 리뷰의 해석 프레임을 정한다. 즉, 본 수치들은 ‘전세계 생성형 AI 사용의 방향성’을 보여주는 강한 신호이지만, 플랫폼 편향(텔레메트리 기반)과 국가별 사용 행태 차이를 완전히 제거한 ‘절대적 진실’로 과잉 해석하지 않아야 한다.

\section{Year in Review}
\subsection{하이라이트 1: 전세계 확산은 상승했지만, 상승의 속도는 고르게 분포하지 않았다}
H2 2025 전세계 AI diffusion은 16.3\%로, H1 2025의 15.1\% 대비 1.2\%p 증가했다. 보고서는 이를 “대략 6명 중 1명”이 생성형 AI 도구를 사용한다고 풀어 설명한다 \href{./archive/web/text/Microsoft-AI-Diffusion-Report-2025-H2.txt}{[1]}. 이 대목의 의미는 단순한 ‘사용자 수 증가’가 아니라, 생성형 AI가 특정 얼리어답터 도구가 아니라 일상적 문제 해결(학습, 업무, 생활 과업)에 동원되는 준-대중 기술로 이동하고 있다는 신호라는 점이다(해석). 다만, 같은 문단에서 곧바로 ‘격차 확대’를 병치함으로써, 이 대중화가 ‘전세계 동시 대중화’가 아니라 ‘선진권에서 더 빠르게 대중화’라는 구조임을 분명히 한다 \href{./archive/web/text/Microsoft-AI-Diffusion-Report-2025-H2.txt}{[1]}.

\subsection{하이라이트 2: Global North–South 격차가 확대되며 “디지털 격차의 재생산”이 관측되었다}
가장 중요한 정량 변화는 지역 격차다. Global North의 H2 2025 사용 비중은 24.7\%, Global South는 14.1\%로 제시되며, H1 대비 증가폭은 각각 +1.8\%p, +1.0\%p다. 결과적으로 격차는 H1 9.8\%p에서 H2 10.6\%p로 커졌다 \href{./archive/web/text/Microsoft-AI-Diffusion-Report-2025-H2.txt}{[1]}. 보고서는 “Global North가 Global South보다 거의 2배 빠르게 성장”했다고 표현한다 \href{./archive/web/text/Microsoft-AI-Diffusion-Report-2025-H2.txt}{[1]}. 이 표현은 속도 차이를 강조하지만, 전략 관점에서는 다음 질문을 남긴다. 격차가 커진 원인이 (1) 비용 및 결제 접근성, (2) 언어·문화 적합성, (3) 제도 및 공공 도입, (4) 네트워크 인프라 등 중 어디에 더 크게 기인하는가. 보고서는 이후 국가 사례와 DeepSeek 사례를 통해 ‘접근성’과 ‘정책/제도’의 중요성을 암시하지만, 기여도 분해는 제공하지 않는다 \href{./archive/web/text/Microsoft-AI-Diffusion-Report-2025-H2.txt}{[1]}.

\subsection{하이라이트 3: 상위권은 견고했고, “순위 안정성” 자체가 구조적 포화 신호로 읽힌다}
상위 30개국의 순위는 전반적으로 안정적이며, 초기부터 디지털 인프라, AI 스킬링, 정부 도입에 투자한 국가들이 계속 선두를 유지한다. 보고서는 UAE, Singapore, Norway, Ireland, France, Spain 등을 대표 예로 든다 \href{./archive/web/text/Microsoft-AI-Diffusion-Report-2025-H2.txt}{[1]}. 실제 표에서 UAE는 H2 2025 64.0\%로 1위를 유지했고, Singapore는 60.9\%로 2위다. 이 둘의 격차가 3\%p 이상으로 벌어졌다고 서술된다 \href{./archive/web/text/Microsoft-AI-Diffusion-Report-2025-H2.txt}{[1]}. 상위권의 안정성은 ‘상위국이 계속 잘한다’는 의미를 넘어서, 동일 지표에서 단기간 내 순위 변동이 어려울 정도로 (1) 성숙한 디지털 생태계, (2) 공공-민간 도입의 일상화, (3) 서비스 접근성(결제, 디바이스, 네트워크)이 이미 충분히 깔려 있음을 반증한다(해석). 반대로 말하면, 후발국이 단기간에 치고 올라오려면 ‘모델 성능’만으로는 부족하고 정책, 제도, 소비자 경험까지 묶인 전환점이 필요하다는 암시가 된다.

\subsection{하이라이트 4: 미국 사례는 “혁신 리더십 = 확산 리더십”이 아님을 보여준다}
보고서는 미국이 AI 인프라와 프론티어 모델 개발에서 선도적임에도, 근로연령 인구 대비 사용 비중 순위는 23위에서 24위로 하락했고 H2 2025 사용률은 28.3\%라고 지적한다 \href{./archive/web/text/Microsoft-AI-Diffusion-Report-2025-H2.txt}{[1]}. 이 관찰은 기술 공급(모델 개발, 클라우드 인프라)과 기술 수요(일반 인구의 실제 사용)가 동일 축이 아니라는 점을 드러낸다. 연구·산업의 중심지일수록 당연히 사용자 비중도 높을 것이라는 직관이 부분적으로 깨지는 지점이며, 특히 “소규모이지만 고도로 디지털화된 경제”가 더 높은 사용 비중을 보인다는 서술은, 확산을 결정하는 변수가 ‘국가 규모’나 ‘혁신 생태계’가 아니라 ‘생활 속 서비스화 정도’일 수 있음을 시사한다(해석) \href{./archive/web/text/Microsoft-AI-Diffusion-Report-2025-H2.txt}{[1]}.

\subsection{하이라이트 5: 대한민국의 급등은 정책·언어 성능·소비자 기능이 결합될 때 확산이 가속된다는 사례로 제시된다}
대한민국은 H2 2025에 25위에서 18위로 7계단 상승했으며, H2 2025 사용률 30.7\%로 제시된다 \href{./archive/web/text/Microsoft-AI-Diffusion-Report-2025-H2.txt}{[1]}. 보고서는 이를 “연말의 가장 분명한 성공 사례”로 묘사하고, 배경을 (1) 국가 정책, (2) 한국어 성능 개선된 프론티어 모델, (3) 대중에게 먹힌 소비자 기능으로 정리한다 \href{./archive/web/text/Microsoft-AI-Diffusion-Report-2025-H2.txt}{[1]}. 특히 별도 섹션에서는 “3개월 만에 극적인 도약”이라 표현하며 생성형 AI 사용이 약 26\%에서 30\% 이상으로 늘고, 2024년 10월 이후 누적 성장이 80\% 이상으로 서술된다 \href{./archive/web/text/Microsoft-AI-Diffusion-Report-2025-H2.txt}{[1]}. 또한 한국어 성능 개선을 설명하기 위해 CSAT(Korean SAT) 벤치마크에서 GPT-3.5 16, GPT-4o 75, GPT-5 100의 점수 변화를 예시로 든다 \href{./archive/web/text/Microsoft-AI-Diffusion-Report-2025-H2.txt}{[1]}. 이 서술은 “언어 적합성(linguistic coverage)이 확산을 견인한다”는 논리를 뒷받침하며, 더 나아가 저자원 언어권에서 향후 모델 성능이 개선될 때 유사한 확산 곡선이 나타날 수 있다는 기대를 덧붙인다 \href{./archive/web/text/Microsoft-AI-Diffusion-Report-2025-H2.txt}{[1]}. 다만 이 부분은 아직 관측 기반의 인과 추정(해석)이며, 언어 성능 향상이 실제 사용 전환을 얼마나 설명하는지에 대한 계량 검증은 보고서 내에서 제공되지 않는다.

\subsection{하이라이트 6: DeepSeek의 부상은 “오픈·무료·배포 전략”이 확산 지형을 재편할 수 있음을 보여준다}
보고서는 2025년의 또 다른 축으로 DeepSeek의 급부상을 든다. 핵심은 모델 가중치를 MIT License로 공개하고, 완전 무료 챗봇을 제공함으로써 비용·기술 장벽을 낮췄다는 점이다 \href{./archive/web/text/Microsoft-AI-Diffusion-Report-2025-H2.txt}{[1]}. 확산 패턴은 North America/Europe에서는 낮고, China, Russia, Iran, Cuba, Belarus와 “아프리카 전역”에서 높다고 서술되며, 아프리카에서의 사용은 다른 지역 대비 2에서 4배로 추정된다고 한다 \href{./archive/web/text/Microsoft-AI-Diffusion-Report-2025-H2.txt}{[1]}. 또한 Huawei 등과의 파트너십을 포함한 배포 전략을 언급하며, 오픈소스 AI가 지정학적 도구로 작동할 수 있음을 시사한다 \href{./archive/web/text/Microsoft-AI-Diffusion-Report-2025-H2.txt}{[1]}. 동시에 표준과 안전(standards and safety)에 대한 질문을 제기한다는 문장으로, ‘확산의 가속’과 ‘거버넌스의 지연’ 사이의 긴장을 정면으로 드러낸다 \href{./archive/web/text/Microsoft-AI-Diffusion-Report-2025-H2.txt}{[1]}.

\section{Cross-Cutting Themes}
\subsection{확산을 설명하는 3요소: 접근성, 적합성, 제도화}
보고서 전반을 관통하는 메시지를 단일 문장으로 요약하면, “AI 확산은 모델의 존재가 아니라, 접근성과 제도적·문화적 적합성이 결합될 때 일어난다”에 가깝다. DeepSeek 사례는 무료 제공, 오픈소스 공개, 지역 배포를 통해 ‘접근성’을 극대화한 경우다 \href{./archive/web/text/Microsoft-AI-Diffusion-Report-2025-H2.txt}{[1]}. 반면 대한민국 사례는 (i) 정책 및 공공 도입의 제도화(AI 전략기구, AI Basic Act 등), (ii) 한국어 성능의 질적 도약, (iii) 이미지 생성 같은 소비자 기능의 바이럴로 ‘적합성’과 ‘경험’이 맞물린 경우로 제시된다 \href{./archive/web/text/Microsoft-AI-Diffusion-Report-2025-H2.txt}{[1]}. UAE 사례는 2017년부터의 선제적 거버넌스 구축과 규제 샌드박스, 인재 비자 등으로 제도화 기반을 누적해 온 방향을 강조한다 \href{./archive/web/text/Microsoft-AI-Diffusion-Report-2025-H2.txt}{[1]}. 서로 다른 경로처럼 보이지만, 결과적으로 (1) 비용과 접근성, (2) 언어/문화/사용성 적합성, (3) 공공-민간의 제도적 흡수라는 세 축 중 적어도 두 축이 동시에 강화될 때 확산이 ‘점프’한다는 공통분모가 읽힌다(해석).

\subsection{‘격차 확대’는 인프라 문제가 아니라, 제품화된 AI의 보급 메커니즘 문제일 수 있다}
Global North와 South 격차가 확대되었다는 사실은 통신망, 디바이스 보급 같은 전통적 디지털 인프라 격차의 연장선으로 설명될 수도 있다. 그러나 보고서가 DeepSeek의 아프리카 확산을 중요한 반례처럼 배치한 점을 보면, 단순히 인프라가 아니라 “어떤 형태로 제품이 배포되는가(무료, 결제, 제약, 제휴)”가 격차를 재구성할 수 있다는 관점을 강조하는 것으로 읽힌다 \href{./archive/web/text/Microsoft-AI-Diffusion-Report-2025-H2.txt}{[1]}. 즉, 동일한 기술이라도 유료 구독형, 특정 결제수단 의존, 지역 차단이 결합되면 확산은 느려지고, 반대로 무료/오픈 배포가 결합되면 기존 질서를 우회할 수 있다. 이때 격차는 ‘접속 격차’가 아니라 ‘사용 가능성(availability) 격차’로 재정의된다(해석).

\subsection{텔레메트리 기반 확산 지표의 전략적 가치와 위험}
AI diffusion은 “기간 내 생성형 AI 제품을 사용한 인구 비중”이라는 직관적 정의를 갖고, OS/디바이스 점유율, 인터넷 보급률, 인구 차이를 보정하여 국가 비교를 시도한다 \href{./archive/web/text/Microsoft-AI-Diffusion-Report-2025-H2.txt}{[1]}. 이는 정책·전략 커뮤니케이션에서 강력한 장점이다. 수치가 단순하고, 반기 단위로 변화 감지가 가능하며, 국가 간 비교가 가능한 프레임을 제공하기 때문이다. 그러나 동시에 이 지표는 특정 사업자(여기서는 Microsoft) 텔레메트리에 기반한다는 점에서, (1) 특정 플랫폼/OS/서비스의 사용자층이 국가별로 다르게 대표될 수 있고, (2) “사용”의 정의(로그인, 요청, 트래픽)가 체감 활용도(업무 내재화, 생산성 향상)와 다를 수 있으며, (3) 정부/기업의 내부 도입(폐쇄망, 프라이빗 모델) 등은 과소 포착될 가능성이 있다. 보고서가 단일 지표의 한계를 먼저 인정하고 추가 지표를 예고한 대목은 이러한 위험을 스스로 관리하려는 장치로 보인다 \href{./archive/web/text/Microsoft-AI-Diffusion-Report-2025-H2.txt}{[1]}.

\subsection{오픈소스 확산과 안전의 긴장: ‘확산의 민주화’는 곧 ‘거버넌스의 분산화’다}
DeepSeek 섹션은 오픈소스가 비용 장벽을 낮추고 후발 지역의 접근성을 높인다고 평가하면서도, 빠른 확산이 standards/safety 이슈를 제기한다고 명시한다 \href{./archive/web/text/Microsoft-AI-Diffusion-Report-2025-H2.txt}{[1]}. 여기서 핵심은 안전의 문제가 단지 모델 품질의 문제가 아니라, 누가 배포를 통제하고, 누가 책임을 지며, 누가 업데이트와 차단을 수행하는지의 문제로 이동한다는 점이다(해석). 폐쇄형 서비스는 중앙 통제와 책임 소재가 비교적 명확하지만, 오픈 배포는 통제의 반경이 넓어지는 대신 접근성은 커진다. 2025년은 이 균형점이 “확산을 통해 체감되기 시작한 해”로 묘사할 수 있다.

\section{Outstanding Questions}
\subsection{AI diffusion이 실제 ‘활용의 질’과 얼마나 정렬되는가}
보고서는 AI diffusion을 사용 경험의 유무 중심으로 정의한다 \href{./archive/web/text/Microsoft-AI-Diffusion-Report-2025-H2.txt}{[1]}. 그러나 전략 관점에서는 “얼마나 많은 사람이 써봤는가”뿐 아니라 “얼마나 깊게 내재화했는가(업무 프로세스, 공공 서비스, 교육 커리큘럼)”가 중요하다. H2 2025의 16.3\%가 생산성/과학발견/학습 성과 같은 중간지표와 어떤 상관을 갖는지, 혹은 특정 국가에서 ‘사용률은 낮아도 고부가가치 활용이 높은’ 패턴이 존재하는지 여부는 본 보고서만으로는 답하기 어렵다.

\subsection{Global North–South 격차 확대의 원인 분해: 비용, 언어, 제도, 규제 중 무엇이 주도했는가}
격차가 9.8\%p에서 10.6\%p로 확대되었다는 관측은 명확하지만, 기여 요인 분해는 제공되지 않는다 \href{./archive/web/text/Microsoft-AI-Diffusion-Report-2025-H2.txt}{[1]}. DeepSeek 사례는 비용과 배포의 영향력을 강하게 시사하고, 한국 사례는 언어 성능과 소비자 기능, 정책의 결합을 보여준다 \href{./archive/web/text/Microsoft-AI-Diffusion-Report-2025-H2.txt}{[1]}. 다만 이 둘이 ‘격차 확대’라는 거시 패턴에 각각 얼마나 기여하는지는 불명확하다. 후속 연구에서는 국가별 가격 접근성(구독료, 결제수단), 언어 성능(벤치마크), 규제 환경(금지/허용/샌드박스), 공공 도입 지수 등을 포함한 다변량 분석이 필요하다(제언).

\subsection{미국의 상대적 순위 하락은 지표의 특성인가, 확산의 구조적 병목인가}
미국은 인프라·프론티어 개발 선도에도 사용 비중 순위가 23위에서 24위로 떨어졌고 사용률은 28.3\%로 제시된다 \href{./archive/web/text/Microsoft-AI-Diffusion-Report-2025-H2.txt}{[1]}. 이는 (1) 국가 규모가 커서 분모가 큰 효과, (2) 특정 인구집단 중심의 활용 편중, (3) 기업 내 프라이빗 활용이 텔레메트리에 덜 잡히는 효과, (4) 실제로 대중적 제품화가 다른 국가보다 덜 진행된 효과 등 여러 가설로 설명될 수 있다. 보고서는 “혁신/인프라만으로는 충분치 않다”는 해석을 붙이지만, 대안 가설을 배제할 증거는 제한적이다 \href{./archive/web/text/Microsoft-AI-Diffusion-Report-2025-H2.txt}{[1]}. 이 지점은 정책적 함의가 큰 만큼, 후속 보고서에서 지표 민감도 분석이 요구된다.

\subsection{DeepSeek의 ‘아프리카 2--4$\times$’ 추정치의 산출 근거와 불확실성}
DeepSeek 사용이 아프리카에서 다른 지역보다 2에서 4배 높게 “estimated”라고 표현되지만, 추정 방식(어떤 데이터 소스, 어떤 기간, 어떤 기준선)이 본문에서는 상세히 제시되지 않는다 \href{./archive/web/text/Microsoft-AI-Diffusion-Report-2025-H2.txt}{[1]}. 특히 텔레메트리 기반 지표가 “생성형 AI 제품 사용”을 포착한다고 했을 때, DeepSeek의 시장 점유는 별도의 측정(앱 점유, 웹 트래픽 등)일 수 있다. 이 두 측정이 어떤 관계인지(동일 데이터 계열인지, 보조 지표인지)가 불명확해, 독자가 수치를 정책 판단에 직접 대입하기 어렵다.

\section{Future Directions}
\subsection{지표 측면: 단일 확산률에서 ‘다중 확산 대시보드’로}
보고서가 “No single metric is perfect”라고 인정하고 추가 지표 보완을 예고한 만큼, 다음 단계는 확산을 다차원으로 분해하는 것이다 \href{./archive/web/text/Microsoft-AI-Diffusion-Report-2025-H2.txt}{[1]}. 예를 들어 (1) 단순 사용 경험(Any-use), (2) 월간/주간 빈도(Active-use), (3) 업무/학습 내재화(Integrated-use), (4) 공공 서비스 적용 범위(Public-use), (5) 개발자 생태계(Open/Dev-use) 같은 지표 묶음이 필요하다. 이렇게 해야 ‘확산 격차’가 실제로는 ‘가벼운 사용 격차’인지 ‘생산적 활용 격차’인지 구별할 수 있다(제언).

\subsection{정책 측면: Global South 확산을 가르는 병목은 ‘접속’이 아니라 ‘구매·사용 가능성’일 수 있다}
DeepSeek의 무료·오픈 전략이 제시하는 함의는, Global South의 확산을 높이기 위해 반드시 최첨단 인프라부터 깔아야 하는 것은 아니라는 점이다(해석). 물론 인프라는 중요하지만, 동시에 결제수단 제약, 구독 비용, 지역 제한, 언어 최적화 부족 같은 “제품화된 장벽”이 확산을 크게 좌우할 수 있다 \href{./archive/web/text/Microsoft-AI-Diffusion-Report-2025-H2.txt}{[1]}. 따라서 정책적 개입도 (1) 공공 영역에서의 무료 접근 채널, (2) 교육용 라이선스 지원, (3) 로컬 언어 성능 개선 투자, (4) 규제 샌드박스 기반의 신뢰 형성 등으로 구체화될 수 있다. UAE의 사례 서술은 샌드박스, 원칙 중심 가이드라인, 인재 비자 등 제도 설계가 신뢰와 확산의 매개가 될 수 있음을 강조한다 \href{./archive/web/text/Microsoft-AI-Diffusion-Report-2025-H2.txt}{[1]}.

\subsection{기술 측면: 언어 성능의 개선을 ‘확산 가속 레버’로 관리하는 방법}
한국 사례는 특정 언어에서 모델 성능이 질적으로 개선될 때 사용이 급증할 수 있음을 보여주는 대표 서사로 구성되어 있다. GPT-4o 및 GPT-5 출시와 함께 CSAT 벤치마크 점수의 큰 폭 상승을 제시하며, 이것이 한국어권 사용성 전환점이었다고 설명한다 \href{./archive/web/text/Microsoft-AI-Diffusion-Report-2025-H2.txt}{[1]}. 향후에는 (1) 언어별 성능 개선 이벤트(모델 릴리스, 파인튜닝)와 (2) 해당 지역의 확산률 변화 사이의 시간 지연(lag)과 크기를 계량적으로 추적해, “언어 성능 투자가 확산을 얼마나 끌어올리는지”를 검증하는 연구가 필요하다(제언). 이는 저자원 언어권에서의 AI 접근성 전략(공공 투자, 데이터 구축, 평가 체계)에 직접 연결된다.

\subsection{거버넌스 측면: 오픈소스 확산의 안전 문제를 ‘속도’에 맞춰 운영할 제도 설계}
보고서가 제기한 standards/safety 이슈는, 오픈소스가 확산의 불균형을 완화할 수 있는 잠재력과 동시에 안전 관리가 더 어려워진다는 딜레마를 드러낸다 \href{./archive/web/text/Microsoft-AI-Diffusion-Report-2025-H2.txt}{[1]}. 따라서 다음 국면의 거버넌스는 “느리지만 완벽한 규제”가 아니라 “빠르게 갱신되는 운영 체계”에 가까워야 한다(해석). 예컨대 (1) 모델/배포 채널별 최소 안전 기준(공개 체크리스트), (2) 고위험 사용처에 대한 경량화된 가이드라인, (3) 침해사례 공유 및 신속 대응을 위한 국제 협업 채널, (4) 오픈 배포 환경에서의 책임 분담(배포자/호스팅/응용 개발자) 정교화 등이 함께 논의되어야 한다. DeepSeek 사례가 던지는 질문은 결국 “확산을 촉진하는 구조를 유지하면서도, 안전의 사각지대를 어떻게 줄일 것인가”로 수렴한다 \href{./archive/web/text/Microsoft-AI-Diffusion-Report-2025-H2.txt}{[1]}.

\section{Appendix}
\subsection{자료 출처 및 재현성(메타데이터)}
이번 런은 단일 URL(=단일 PDF) 1건을 대상으로 수집되었다. 출처 URL은 https://www.microsoft.com/en-us/research/wp-content/uploads/2026/01/Microsoft-AI-Diffusion-Report-2025-H2.pdf 이며, 수집 결과물은 PDF 1건, PDF 텍스트 추출 1건, Tavily extract 1건으로 구성된다 \href{./archive/20260112\_ms-ai-diffusion-index.md}{[2]}. 실행 파라미터는 `--days 30 --max-results 8 --download-pdf --openalex` 옵션이었으나 실제 입력은 URL 1개만 존재했고 queries는 없었다 \href{./archive/20260112\_ms-ai-diffusion-index.md}{[2]}, \href{./archive/\_job.json}{[5]}.

\subsection{지표 정의(보고서 원문 기준)}
보고서는 AI diffusion을 “해당 기간 동안 생성형 AI 제품을 사용한 전세계 인구 비중”으로 정의하며, 집계·익명화된 Microsoft 텔레메트리를 기반으로 산출한 뒤 OS/디바이스 시장점유율, 인터넷 보급률, 국가 인구 차이를 반영하도록 보정한다고 설명한다. 또한 단일 지표의 한계(“No single metric is perfect”)를 명시하고 향후 추가 지표로 보완하겠다고 밝힌다 \href{./archive/web/text/Microsoft-AI-Diffusion-Report-2025-H2.txt}{[1]}.

\subsection{커버리지/파일 경로 관련 주의}
scout notes에는 PDF가 존재한다고 되어 있으나, 이번 환경에서 `list_archive_files`로는 `./archive/web/pdf/*`, `./archive/web/text/*`, `./archive/tavily_extract/*` 목록이 비어 있는 것으로 반환되는 등 경로 기준 불일치 가능성이 있었다. 다만 `read_document`로는 파일들이 정상적으로 읽혀 콘텐츠가 확인되었으며, 본 리뷰의 근거 인용은 텍스트 파일 읽기 결과를 기준으로 수행 가능하다 \href{./archive/web/text/Microsoft-AI-Diffusion-Report-2025-H2.txt}{[1]}, \href{./archive/tavily\_extract/0001\_https\_www.microsoft.com\_en-us\_research\_wp-content\_uploads\_2026\_01\_Microsoft-AI-Diffusion-Report-2025-H2.pdf.txt}{[4]}.

\section*{Figures}
\paragraph{Figures referenced.} Figure~\ref{fig:1}: Source PDF: Microsoft-AI-Diffusion-Report-2025-H2.pdf Figure~\ref{fig:2}: Source PDF: Microsoft-AI-Diffusion-Report-2025-H2.pdf Figure~\ref{fig:3}: Source PDF: Microsoft-AI-Diffusion-Report-2025-H2.pdf Figure~\ref{fig:4}: Source PDF: Microsoft-AI-Diffusion-Report-2025-H2.pdf

\begin{figure}[htbp]
\centering
\includegraphics[width=\linewidth]{report\_assets/figures/.\_archive\_web\_pdf\_Microsoft-AI-Diffusion-Report-2025-H2.pdf-7287640b.jpeg}
\caption{Source: \\texttt{./archive/web/pdf/Microsoft-AI-Diffusion-Report-2025-H2.pdf}, page 1.}
\label{fig:1}
\end{figure}

\begin{figure}[htbp]
\centering
\includegraphics[width=\linewidth]{report\_assets/figures/.\_archive\_web\_pdf\_Microsoft-AI-Diffusion-Report-2025-H2.pdf-9c8cb5d4.png}
\caption{Source: \\texttt{./archive/web/pdf/Microsoft-AI-Diffusion-Report-2025-H2.pdf}, page 1.}
\label{fig:2}
\end{figure}

\begin{figure}[htbp]
\centering
\includegraphics[width=\linewidth]{report\_assets/figures/.\_archive\_web\_pdf\_Microsoft-AI-Diffusion-Report-2025-H2.pdf-46e6962e.png}
\caption{Source: \\texttt{./archive/web/pdf/Microsoft-AI-Diffusion-Report-2025-H2.pdf}, page 2.}
\label{fig:3}
\end{figure}

\begin{figure}[htbp]
\centering
\includegraphics[width=\linewidth]{report\_assets/figures/.\_archive\_web\_pdf\_Microsoft-AI-Diffusion-Report-2025-H2.pdf-dcda6a48.png}
\caption{Source: \\texttt{./archive/web/pdf/Microsoft-AI-Diffusion-Report-2025-H2.pdf}, page 18.}
\label{fig:4}
\end{figure}

\section*{Report Prompt}
\begin{verbatim}
MS 에서 발간한 AI 확산 보고서에 대해서 확인해주고 보고서를 작성해줘.
영문으로 되어있는 내용에 대해 한글로 최대한 심도있게 보고서를 번역한다고 생각하고.

문체는 설명형/서술형 리뷰로, 너무 딱딱한 학술문체를 피하고 자연스러운 연결 문장과 전환을 사용.
문장 길이를 섞되, 요약-근거-해석 흐름이 읽히게 작성. 필요시 figure나 그림을 추출하여 리포트에 사용함.
\end{verbatim}
\section*{References}
\renewcommand{\labelenumi}{[\arabic{enumi}]}
\begin{enumerate}
\item Microsoft-AI-Diffusion-Report-2025-H2.txt --- \href{./archive/web/text/Microsoft-AI-Diffusion-Report-2025-H2.txt}{\texttt{./archive/web/text/Microsoft-AI-Diffusion-Report-2025-H2.txt}}
\item 20260112\_ms-ai-diffusion-index.md --- \href{./archive/20260112\_ms-ai-diffusion-index.md}{\texttt{./archive/20260112\_ms-ai-diffusion-index.md}}
\item 20260112\_ms-ai-diffusion.txt --- \href{./instruction/20260112\_ms-ai-diffusion.txt}{\texttt{./instruction/20260112\_ms-ai-diffusion.txt}}
\item 0001\_https\_www.microsoft.com\_en-us\_research\_wp-content\_uploads\_2026\_01\_Microsoft-AI-Diffusion-Report-2025-H2.pdf.txt --- \href{./archive/tavily\_extract/0001\_https\_www.microsoft.com\_en-us\_research\_wp-content\_uploads\_2026\_01\_Microsoft-AI-Diffusion-Report-2025-H2.pdf.txt}{\texttt{./archive/tavily\_extract/0001\_https\_www.microsoft.com\_en-us\_research\_wp-content\_uploads\_2026\_01\_Microsoft-AI-Diffusion-Report-2025-H2.pdf.txt}}
\item \_job.json --- \href{./archive/\_job.json}{\texttt{./archive/\_job.json}}
\end{enumerate}
\section*{Miscellaneous}
\small
\begin{itemize}
\item Generated at: 2026-01-14 21:55:23
\item Duration: 00:10:32 (632.09s)
\item Model: gpt-5.2
\item Quality strategy: none
\item Quality iterations: 0
\item Template: annual\_review
\item Output format: tex
\item PDF compile: enabled
\end{itemize}
\normalsize
\end{document}
