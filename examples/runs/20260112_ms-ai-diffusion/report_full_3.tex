\documentclass[11pt]{article}
\usepackage{kotex}
\usepackage[margin=1in]{geometry}
\usepackage{hyperref}
\usepackage{amsmath,amssymb}
\usepackage{graphicx}
\usepackage{booktabs}
\usepackage{enumitem}
\title{ Federlicht Report - 20260112\_ms-ai-diffusion }
\author{ Hyun-Jung Kim / AI Governance Team }
\date{ 2026-01-14 }
\begin{document}
\maketitle

\noindent\textit{Federlicht assisted and prompted by "Hyun-Jung Kim / AI Governance Team" — 2026-01-14 23:40}

\section{Abstract}
Microsoft AI Economy Institute가 2026년 1월 공개한 “Global AI Adoption in 2025 — A Widening Digital Divide”는 2025년 상반기(H1) 대비 하반기(H2)의 생성형 AI 사용 확산이 전 세계적으로 증가했지만, 그 증가가 지역 간 불균등하게 나타나며 ‘디지털 격차’가 오히려 확대되었음을 강조한다. 핵심 지표는 전 세계 AI diffusion(해당 기간 생성형 AI 제품 사용 비중)이 15.1\%에서 16.3\%로 1.2\%p 상승한 반면, Global North는 22.9\%에서 24.7\%(+1.8\%p), Global South는 13.1\%에서 14.1\%(+1.0\%p)로 증가폭이 달라 격차가 9.8\%p에서 10.6\%p로 커졌다는 점이다. 지표는 “aggregated and anonymized Microsoft telemetry”에 기반하고, OS 및 디바이스 시장점유율, 인터넷 보급, 국가 인구 차이를 반영해 조정했다고 설명한다. 본 리뷰는 단일 1차 자료(PDF 1개)에 대한 설명형 번역 리뷰로서, 해당 문서가 제시하는 수치, 사례(한국, UAE, DeepSeek), 그리고 정책적 함의를 중심으로 2025년 하반기의 변화 궤적을 통합적으로 정리한다. \href{./archive/web/text/Microsoft-AI-Diffusion-Report-2025-H2.txt}{[1]} \href{https://www.microsoft.com/en-us/research/wp-content/uploads/2026/01/Microsoft-AI-Diffusion-Report-2025-H2.pdf}{[2]}

\section{Introduction}
\subsection{범위와 목표}
본 보고서는 Microsoft Research 산하 Microsoft AI Economy Institute의 확산 측정 결과를 바탕으로, 2025년 생성형 AI 사용이 어떤 속도로, 어떤 지역 및 국가에서, 어떤 조건(정책, 신뢰, 접근성, 오픈소스 전략) 아래 확산되었는지 설명형으로 정리한다. 독자는 연구자 및 전략가를 상정하며, 단순 요약을 넘어 “수치(무엇이 바뀌었나)–근거(무엇을 근거로 말하나)–해석(왜 중요한가)”의 흐름을 유지한다.

\subsection{자료 선정 기준과 시간 창}
이번 런(run)은 입력 URL 1개에 기반한 단일 문서 아카이브로 구성되어 있다. 따라서 본 리뷰의 1차 근거는 해당 보고서 PDF와 그 텍스트 추출본에 한정된다. \href{./archive/\_job.json}{[3]} \href{./archive/20260112\_ms-ai-diffusion-index.md}{[4]} 분석 시간 창은 보고서가 비교 대상으로 삼은 2025년 상반기(H1)와 하반기(H2)이다. \href{./archive/web/text/Microsoft-AI-Diffusion-Report-2025-H2.txt}{[1]}

\subsection{핵심 개념: AI diffusion의 의미와 측정 프레임}
보고서는 AI diffusion을 “reported period 동안 생성형 AI 제품을 사용한 사람의 비중”으로 정의한다. 즉 ‘모델 성능’이나 ‘기업 투자’가 아니라, 인구 수준에서의 사용 경험 비율(usage share)을 직접 추적하려는 시도다. 산출은 Microsoft 텔레메트리를 집계·익명화한 뒤, 국가별로 OS 및 디바이스 시장점유율, 인터넷 보급률, 인구 규모 차이를 반영해 조정했다고 밝힌다. 동시에 “어떤 단일 지표도 완벽하지 않다”고 전제하며, 향후 추가 지표로 보완하겠다고 명시한다. 이 전제는 해석에 중요하다. 이 보고서가 강한 메시지를 갖더라도, 그것이 ‘유일한 정답’이 아니라 ‘현재 가능한 최선의 비교 지표 중 하나’임을 스스로 한정하고 있기 때문이다. \href{./archive/web/text/Microsoft-AI-Diffusion-Report-2025-H2.txt}{[1]}

\section{Year in Review}
\subsection{하이라이트 1: 전 세계 사용은 늘었지만, “1/6”의 시대는 여전히 불균등하게 열린다}
보고서는 2025년 하반기 전 세계 AI diffusion이 16.3\%로 올라 “대략 6명 중 1명”이 생성형 AI 도구를 사용한다고 요약한다. 상반기 15.1\%에서 1.2\%p 증가한 수치다. 기술이 ‘대중화’ 국면으로 넘어가고 있다는 신호로 읽히지만, 이 증가가 글로벌 평균의 따뜻한 숫자에 그치지 않고 지역별로 다른 속도를 보였다는 점이 보고서의 핵심 문제의식이다. \href{./archive/web/text/Microsoft-AI-Diffusion-Report-2025-H2.txt}{[1]}

\subsection{하이라이트 2: Global North–South 격차는 축소가 아니라 확대(9.8\%p→10.6\%p)}
하반기 데이터가 보여주는 가장 직접적인 결론은 격차의 확대다. Global North는 22.9\%에서 24.7\%로 1.8\%p 늘었고, Global South는 13.1\%에서 14.1\%로 1.0\%p 늘었다. 증가율의 차이 때문에 격차가 9.8\%p에서 10.6\%p로 커졌다. 보고서는 이를 “Global North가 Global South보다 거의 두 배 빠르게 성장했다”고 문장으로도 강조한다. 요컨대, 생성형 AI는 확산되고 있지만 ‘확산의 기울기’가 지역별로 다르며, 그 차이가 누적되면서 디지털 격차를 좁히기보다 넓히는 방향으로 작동하고 있다는 진단이다. \href{./archive/web/text/Microsoft-AI-Diffusion-Report-2025-H2.txt}{[1]}

\subsection{하이라이트 3: 상위권 국가는 ‘안정적’이며, 조기 인프라 투자가 계속 보상을 받는다}
Top 30 국가의 전반적 순위는 “steady(안정적)”라고 서술된다. 즉 상위권은 쉽게 바뀌지 않는다. 보고서는 UAE, Singapore, Norway, Ireland, France, Spain 등 디지털 인프라, AI 스킬링, 정부 도입을 일찍부터 추진한 국가들이 선도한다고 설명한다. 특히 UAE는 하반기 64.0\%로 1위를 유지하며 상반기 59.4\%에서 더 상승했고, Singapore(60.9\%)와 3\%p 이상 격차를 벌렸다고 언급한다. 이 대목에서 보고서가 전달하는 암묵적 메시지는 분명하다. “사용률(확산)”은 단기간의 유행이나 모델 출시만으로 결정되지 않으며, 정책·인프라·역량 구축의 누적 효과가 상위권을 고착화한다는 것이다. \href{./archive/web/text/Microsoft-AI-Diffusion-Report-2025-H2.txt}{[1]}

\subsection{하이라이트 4: 미국의 역설(강한 혁신 리더십이 ‘높은 사용 비중’을 자동 보장하지 않는다)}
보고서는 미국이 AI 인프라와 프론티어 모델 개발을 주도한다고 인정하면서도, 사용률 기준 순위는 23위에서 24위로 하락했고 하반기 사용률은 28.3\%라고 적시한다. 여기서 흥미로운 지점은 ‘혁신 리더십’과 ‘대중적 확산’ 사이에 간극이 있음을 보고서가 직접 강조한다는 점이다. 즉, 연구개발과 공급(모델 개발·인프라) 측면의 선도와, 인구 대비 실제 사용(채택) 사이에는 별도의 병목이 존재할 수 있다는 관찰이다. 전략적으로는 “모델을 만든 나라”와 “가장 넓게 쓰는 나라”가 반드시 일치하지 않는다는 경고로도 읽힌다. \href{./archive/web/text/Microsoft-AI-Diffusion-Report-2025-H2.txt}{[1]}

\subsection{하이라이트 5: 한국의 급상승은 ‘정책–언어 성능–소비자 기능’의 결합으로 설명된다}
보고서는 한국을 “가장 분명한 연말 성공 사례(end-of-year success story)”로 제시한다. 하반기 동안 한국은 25위에서 18위로 7계단 상승했고, 생성형 AI 사용이 약 26\%에서 30\%를 넘었다고 말한다. 더 나아가 “2024년 10월 이후 누적 성장(total growth)이 80\% 이상”이며, 이는 글로벌 평균(35\%)과 미국(25\%)보다 빠르다고 비교한다. 또한 “한국이 미국에 이어 세계 2위의 ChatGPT 유료 구독 시장”이라는 진술도 포함되어 있다. 다만 이 문장은 보고서가 인용한 외부 기사([4])를 매개로 들어온 것이므로, 본 리뷰에서도 ‘보고서의 주장’으로 귀속시키는 것이 적절하다. \href{./archive/web/text/Microsoft-AI-Diffusion-Report-2025-H2.txt}{[1]}

보고서가 제시하는 한국 확산의 동인은 세 가지다. 첫째, 국가 정책이 AI 통합을 가속했다는 점이다. 예로 정부가 국가 AI 조정 기구를 National AI Strategy Committee로 재구성하고, 혁신과 거버넌스 균형을 목표로 하는 AI Basic Act를 제정했다고 언급한다. 둘째, 한국어에서의 프론티어 모델 성능이 급격히 좋아졌다는 점이다. OpenAI GPT-4o(2025년 4월)와 GPT-5(2025년 8월) 출시가 한국어 사용 경험을 ‘실용적이고 신뢰할 만한 수준’으로 끌어올렸다고 서술하며, Korean SAT(CSAT) 벤치마크 점수가 GPT-3.5 16, GPT-4o 75, GPT-5 100으로 “dramatically” 상승했다고 제시한다. 셋째, 소비자 기능이 대중에게 먹혔다는 점이다. 특히 2025년 4월 한국 소셜 플랫폼에서 “Ghibli-style images”가 바이럴되며 비기술 사용자 다수를 첫 사용자로 유입시켰고, 이후에도 다른 AI 기능을 계속 탐색하게 만들었다는 내러티브를 제공한다. 이 구성은 한국의 확산이 ‘단일 요인’이 아니라, 제도적 기반과 제품 경험(언어 성능) 그리고 문화적 촉매가 맞물린 결과임을 강조하기 위한 장치로 보인다. \href{./archive/web/text/Microsoft-AI-Diffusion-Report-2025-H2.txt}{[1]}

\subsection{하이라이트 6: DeepSeek의 부상은 ‘접근성’이 확산의 지형을 다시 그릴 수 있음을 보여준다}
보고서는 2025년의 병렬적 변화로 DeepSeek의 급부상을 든다. DeepSeek를 “open-source AI platform”으로 규정하면서, 모델 가중치를 MIT License로 공개했고 무료 챗봇을 제공함으로써 비용·기술 장벽을 제거했다고 설명한다. 그 결과 채택이 중국, 러시아, 이란, 쿠바, 벨라루스 등과 아프리카에서 강하게 나타났으며, 아프리카에서는 DeepSeek 사용이 다른 지역보다 2~4배 높게 추정된다고 말한다. 또한 Huawei 같은 기업과의 파트너십, 통신 서비스 통합 등을 통해 “strategically distributed”되었고, 오픈소스 AI가 지정학적 도구(geopolitical instrument)로 기능할 수 있다고 지적한다. 이 부분은 ‘모델 품질 경쟁’만으로는 설명되지 않는 확산 동학, 즉 접근성(무료, 제약이 적은 배포), 현지 유통망, 정치·규제 환경이 실제 사용을 크게 좌우한다는 논지를 구성한다. \href{./archive/web/text/Microsoft-AI-Diffusion-Report-2025-H2.txt}{[1]}

\section{Cross-Cutting Themes}
\subsection{테마 1: 확산(adoption)은 성능의 함수이면서, 동시에 제도의 함수다}
보고서의 여러 사례를 관통하면 “좋은 모델이 나오면 쓴다”는 단선적 설명을 넘어서게 된다. 한국 사례에서 모델 성능(특히 언어 성능)은 분명 중요한 변곡점으로 제시되지만, 동시에 정부의 조정 기구, 법제(AI Basic Act), 교육·인재 파이프라인 확대 등 제도적 기반이 병렬로 서술된다. 반대로 미국 사례는 혁신·인프라 리더십이 있어도 인구 대비 사용 비중이 상위권으로 자동 상승하지 않는다는 점을 보여주며, 제도·현장 도입·대중 신뢰 같은 ‘사회적 전달 체계’가 별도의 축임을 암시한다. \href{./archive/web/text/Microsoft-AI-Diffusion-Report-2025-H2.txt}{[1]}

\subsection{테마 2: 신뢰(trust)는 ‘사용률’의 숨은 가속 페달로 다뤄진다}
UAE 대목에서 보고서는 신뢰를 거의 핵심 지표처럼 호출한다. 2017년 세계 최초의 AI 담당 장관 임명, 국가 전략, 규제 샌드박스, 인재 비자, 원칙 중심 가이드라인 등을 “Regulatory pragmatism”으로 묶고, 그 결과가 ‘신뢰’로 나타난다는 논리 구조를 갖는다. 또한 2025 Edelman Trust Barometer를 인용해 UAE의 AI trust가 약 67\%, 미국은 32\%라고 대비시키며 35\%p 격차를 강조한다. 이 수치는 본 런 범위 밖의 외부 문서를 보고서가 인용한 것이므로, 독자는 원문 확인이 필요하다는 점을 염두에 둬야 한다. 그럼에도 보고서 내 내러티브는 명확하다. 사회적 수용과 신뢰가 충분히 높으면, 기술이 일상 거래(daily transactions)로 들어오는 속도가 달라질 수 있다는 것이다. \href{./archive/web/text/Microsoft-AI-Diffusion-Report-2025-H2.txt}{[1]}

\subsection{테마 3: ‘접근성’은 기술 확산의 지리적 경로를 바꾼다}
DeepSeek 사례는 접근성이 확산의 지리를 재배치할 수 있음을 보여주는 장으로 쓰인다. 무료, 오픈소스, 전략적 유통이라는 조합이 기존 서구 플랫폼이 약하거나 제약을 받는 시장에서 강한 견인력을 만들었다는 주장이다. 보고서는 이를 미중 경쟁의 한 국면(각자의 “national models” 채택을 둘러싼 경쟁)으로까지 확장해 해석한다. 이 대목의 실증은 보고서 내 텔레메트리 기반 확산 지표와, 외부 참고문헌(예: Bloomberg 등)에 의존하는 서술이 섞여 있다. 따라서 ‘추세 제시’로는 유효하되, 지정학적 결론을 정책으로 옮기려면 추가 검증이 필요하다는 점도 함께 읽어야 한다. \href{./archive/web/text/Microsoft-AI-Diffusion-Report-2025-H2.txt}{[1]}

\subsection{테마 4: 상위권의 고착과 ‘격차의 관성’}
Top 30의 순위가 대체로 안정적이라는 서술은, 격차가 단지 ‘현재 상태’가 아니라 ‘관성(inertia)’을 가진 구조라는 함의를 낳는다. 인프라, 교육, 정부 도입을 일찍 시작한 국가들이 계속 선도하는 구조에서, 후발 국가가 단기간에 추격하기는 어렵다. 보고서가 Global North–South 격차 확대를 핵심 메시지로 내세우는 것도, 이런 관성이 지속될 경우 격차가 자동으로 줄어들지 않음을 시사하기 때문이다. \href{./archive/web/text/Microsoft-AI-Diffusion-Report-2025-H2.txt}{[1]}

\section{Outstanding Questions}
\subsection{질문 1: 텔레메트리 기반 지표가 ‘실제 사용’의 어떤 부분을 놓칠 수 있는가}
보고서는 지표가 Microsoft 텔레메트리에서 나오며 여러 국가 요인을 조정한다고 설명하지만, 여전히 플랫폼 편향(어떤 생성형 AI 제품이 관측 가능한가), 사용의 깊이(가벼운 체험 vs 업무 통합), 다중 계정/공유 디바이스, 기업 내 폐쇄형 사용 등은 충분히 포착되지 않을 수 있다. 보고서도 “No single metric is perfect”라고 인정한다. 따라서 diffusion을 정책 목표로 삼으려면, (1) 사용 빈도, (2) 생산성/학습 성과, (3) 산업별 도입, (4) 공공서비스 내 내재화 같은 보조 지표가 병행되어야 한다. \href{./archive/web/text/Microsoft-AI-Diffusion-Report-2025-H2.txt}{[1]}

\subsection{질문 2: ‘Global North/Global South’ 구분이 설명하는 것과 숨기는 것}
Global North와 South는 큰 방향성을 보여주지만, 같은 South 안에서도 국가별로 인터넷 보급, 교육, 언어 자원, 규제 환경, 결제 인프라가 매우 다르다. 보고서 부록은 국가별 H1/H2 및 변화량을 제공하지만, 본문 논의는 주로 대지역 평균과 상위권 사례에 집중한다. 정책적 처방을 위해서는 South 내부의 이질성을 더 세분화해 어떤 조건에서 확산이 가속되는지 확인할 필요가 있다. \href{./archive/web/text/Microsoft-AI-Diffusion-Report-2025-H2.txt}{[1]}

\subsection{질문 3: 한국 사례의 일반화 가능성(언어 성능 개선이 항상 ‘대중 확산’으로 이어지는가)}
보고서는 한국어 모델 성능의 급상승과 대중 확산을 강하게 연결한다. 그러나 이것이 다른 언어권에서도 동일하게 재현되는지는 별개의 문제다. 한국은 디지털 인프라, 교육 수준, 강한 모바일 생태계 등 기저 조건이 이미 높고, 여기에 ‘바이럴한 소비자 기능’이 결합했다는 설명이 함께 붙는다. 즉 성능 개선이 필요조건일 수는 있어도 충분조건인지는 불명확하다. 보고서의 서술을 정책으로 번역하려면, 언어 성능 개선, 제품화, 유통 채널, 문화적 촉매가 각각 어느 정도의 기여를 했는지 분해하는 후속 분석이 필요하다. \href{./archive/web/text/Microsoft-AI-Diffusion-Report-2025-H2.txt}{[1]}

\subsection{질문 4: DeepSeek 확산의 계량적 근거와 안전성의 긴장}
보고서는 아프리카에서 DeepSeek 사용이 2~4배 높다고 추정하며, 오픈소스 확산이 안전·표준 문제를 제기한다고 말한다. 그러나 ‘사용 증가’와 ‘안전 리스크’ 사이의 실제 상관(예: 악용, 허위정보, 프라이버시 침해, 현지 규제 역량)을 어떤 데이터로 연결할 것인지가 남는다. 오픈소스 확산을 단지 지정학적 도구로만 해석할 경우, 개발자 생태계의 혁신·현지화 이점이 과소평가될 수 있고, 반대로 접근성만 강조할 경우 안전의 외부비용이 과소평가될 수 있다. 양자를 동시에 계량화하는 프레임이 필요하다. \href{./archive/web/text/Microsoft-AI-Diffusion-Report-2025-H2.txt}{[1]}

\section{Future Directions}
\subsection{방향 1: 확산 지표의 다층화(share에서 depth로)}
보고서의 diffusion은 “사용한 적이 있는 사람의 비중”에 가깝다. 다음 단계는 사용의 ‘깊이’를 포함하는 것이다. 예를 들어 (1) 주간/월간 반복 사용, (2) 업무 도구 체인에 통합된 사용, (3) 교육 현장에서의 커리큘럼 내재화, (4) 공공서비스 프로세스 자동화에 실제로 기여하는 사용 등으로 확장해야 한다. 보고서도 장차 추가 지표로 보완하겠다고 밝힌 만큼, 연례 업데이트에서 어떤 보조 지표가 붙는지가 전략적 관전 포인트다. \href{./archive/web/text/Microsoft-AI-Diffusion-Report-2025-H2.txt}{[1]}

\subsection{방향 2: ‘격차 축소’는 보급률만이 아니라 결제·접근·신뢰 인프라의 문제로 다뤄져야 한다}
Global South의 확산 속도를 끌어올리려면 단순 교육이나 캠페인 이상의 접근이 필요하다. 보고서가 DeepSeek의 무료·오픈소스 모델을 ‘장벽 제거’로 설명하듯, 결제 수단, 가격 정책, 로컬 언어 지원, 통신/디바이스 번들링 같은 유통 레버가 확산에 직접적 영향을 미칠 수 있다. 동시에 UAE 사례가 보여주듯, 신뢰를 구축하는 규제 실험장(샌드박스), 원칙 중심 가이드라인, 작동하는 공공서비스 레퍼런스가 확산의 사회적 토대를 만들 수 있다. \href{./archive/web/text/Microsoft-AI-Diffusion-Report-2025-H2.txt}{[1]}

\subsection{방향 3: 국가 전략은 ‘모델 개발’과 ‘대중 채택’의 분업을 전제로 다시 설계될 수 있다}
미국 사례의 역설은, 최상위 혁신 역량을 가진 국가도 인구 대비 채택은 상대적으로 낮을 수 있음을 보여준다. 이는 정책이 (1) 프론티어 R\&D, (2) 제품화 및 중소기업 도입, (3) 공공부문 적용, (4) 교육·스킬링, (5) 사회적 신뢰 구축을 분리된 트랙으로 운영할 필요가 있음을 시사한다. 반대로 한국 사례는 이 트랙들이 동시에 움직일 때 단기간에도 순위 상승이 가능하다는 ‘결합 모델’을 제시한다. \href{./archive/web/text/Microsoft-AI-Diffusion-Report-2025-H2.txt}{[1]}

\subsection{방향 4: 오픈소스 확산을 둘러싼 거버넌스 실험(안전, 표준, 책임의 재배치)}
DeepSeek 사례가 던지는 질문은 단순히 “누가 더 많이 쓰게 하느냐”가 아니다. 오픈소스 모델이 빠르게 전파될 때 안전과 책임은 어디에 놓이는가, 표준은 어떻게 합의되는가가 핵심이 된다. 보고서가 ‘감독 제한(limited oversight)’을 우려한 만큼, 2026년의 과제는 (1) 오픈소스 모델의 안전 가이드라인, (2) 배포 채널(통신사, 앱마켓, 기업 솔루션)에서의 책임 배분, (3) 현지 규제 역량 강화와 같은 제도 실험으로 이어질 가능성이 크다. \href{./archive/web/text/Microsoft-AI-Diffusion-Report-2025-H2.txt}{[1]}

\section{Appendix}
\subsection{A. 본 리뷰에서 활용한 핵심 수치(보고서 원문 인용)}
\begin{itemize}
\item World AI diffusion: H1 2025 15.10\% $\rightarrow$ H2 2025 16.30\% (Change +1.2\%p). \href{./archive/web/text/Microsoft-AI-Diffusion-Report-2025-H2.txt}{[1]}
\item Global North: 22.90\% $\rightarrow$ 24.70\% (+1.8\%p); Global South: 13.10\% $\rightarrow$ 14.10\% (+1.0\%p). \href{./archive/web/text/Microsoft-AI-Diffusion-Report-2025-H2.txt}{[1]}
\item Gap: 9.8\%p(H1) $\rightarrow$ 10.6\%p(H2). \href{./archive/web/text/Microsoft-AI-Diffusion-Report-2025-H2.txt}{[1]}
\item Top 국가(예시): UAE H2 64.0\%, Singapore 60.9\% 등. \href{./archive/web/text/Microsoft-AI-Diffusion-Report-2025-H2.txt}{[1]}
\item U.S.: H2 usage rate 28.3\%, rank 23rd $\rightarrow$ 24th. \href{./archive/web/text/Microsoft-AI-Diffusion-Report-2025-H2.txt}{[1]}
\item South Korea: 25th $\rightarrow$ 18th, H2 30.7\% (H1 25.9\% 대비 +4.8\%p). \href{./archive/web/text/Microsoft-AI-Diffusion-Report-2025-H2.txt}{[1]}
\item (부록 표 발췌 예시) Japan: 16.7\% $\rightarrow$ 19.1\% (+2.4\%p); China: 15.4\% $\rightarrow$ 16.3\% (+0.9\%p); India: 14.2\% $\rightarrow$ 15.7\% (+1.4\%p); Vietnam: 21.2\% $\rightarrow$ 23.5\% (+2.3\%p). \href{./archive/web/text/Microsoft-AI-Diffusion-Report-2025-H2.txt}{[1]}
\end{itemize}

\subsection{B. 도표(텍스트 추출본 기준)로 확인되는 ‘그림 대용’ 근거}
본 런에서는 PDF 이미지 자체를 추출해 삽입하지 않고, 텍스트 추출본에 포함된 도표 내용을 ‘그림 대용’으로 정리한다.
\begin{itemize}
\item Executive Summary 내 표(Region별 H1/H2/Change) 및 “AI Diffusion by Economy H2 2025” 지도형 범례(사용 비중 구간). \href{./archive/web/text/Microsoft-AI-Diffusion-Report-2025-H2.txt}{[1]}
\item Top 30 국가의 H1 vs H2 막대/순위(국가명, 퍼센트, 변화 폭이 함께 제시). \href{./archive/web/text/Microsoft-AI-Diffusion-Report-2025-H2.txt}{[1]}
\item Global North vs Global South 비교 막대(13.1$\rightarrow$14.1, 22.9$\rightarrow$24.7) 및 격차…

\section*{Figures}
\paragraph{Figures referenced.} Figure~\ref{fig:1}: Source PDF: Microsoft-AI-Diffusion-Report-2025-H2.pdf Figure~\ref{fig:2}: Source PDF: Microsoft-AI-Diffusion-Report-2025-H2.pdf Figure~\ref{fig:3}: Source PDF: Microsoft-AI-Diffusion-Report-2025-H2.pdf Figure~\ref{fig:4}: Source PDF: Microsoft-AI-Diffusion-Report-2025-H2.pdf

\begin{figure}[htbp]
\centering
\includegraphics[width=\linewidth]{report\_assets/figures/.\_archive\_web\_pdf\_Microsoft-AI-Diffusion-Report-2025-H2.pdf-7287640b.jpeg}
\caption{Source: \\texttt{./archive/web/pdf/Microsoft-AI-Diffusion-Report-2025-H2.pdf}, page 1.}
\label{fig:1}
\end{figure}

\begin{figure}[htbp]
\centering
\includegraphics[width=\linewidth]{report\_assets/figures/.\_archive\_web\_pdf\_Microsoft-AI-Diffusion-Report-2025-H2.pdf-9c8cb5d4.png}
\caption{Source: \\texttt{./archive/web/pdf/Microsoft-AI-Diffusion-Report-2025-H2.pdf}, page 1.}
\label{fig:2}
\end{figure}

\begin{figure}[htbp]
\centering
\includegraphics[width=\linewidth]{report\_assets/figures/.\_archive\_web\_pdf\_Microsoft-AI-Diffusion-Report-2025-H2.pdf-46e6962e.png}
\caption{Source: \\texttt{./archive/web/pdf/Microsoft-AI-Diffusion-Report-2025-H2.pdf}, page 2.}
\label{fig:3}
\end{figure}

\begin{figure}[htbp]
\centering
\includegraphics[width=\linewidth]{report\_assets/figures/.\_archive\_web\_pdf\_Microsoft-AI-Diffusion-Report-2025-H2.pdf-dcda6a48.png}
\caption{Source: \\texttt{./archive/web/pdf/Microsoft-AI-Diffusion-Report-2025-H2.pdf}, page 18.}
\label{fig:4}
\end{figure}

\section*{Report Prompt}
\begin{verbatim}
MS 에서 발간한 AI 확산 보고서에 대해서 확인해주고 보고서를 작성해줘.
영문으로 되어있는 내용에 대해 한글로 최대한 심도있게 보고서를 번역한다고 생각하고.

문체는 설명형/서술형 리뷰로, 너무 딱딱한 학술문체를 피하고 자연스러운 연결 문장과 전환을 사용.
문장 길이를 섞되, 요약-근거-해석 흐름이 읽히게 작성. 필요시 figure나 그림을 추출하여 리포트에 사용함.
\end{verbatim}
\section*{References}
\renewcommand{\labelenumi}{[\arabic{enumi}]}
\begin{enumerate}
\item Microsoft-AI-Diffusion-Report-2025-H2.txt --- \href{./archive/web/text/Microsoft-AI-Diffusion-Report-2025-H2.txt}{\texttt{./archive/web/text/Microsoft-AI-Diffusion-Report-2025-H2.txt}}
\item www.microsoft.com/en-us/research/wp-content/uploads/2026/01/M... --- \href{https://www.microsoft.com/en-us/research/wp-content/uploads/2026/01/Microsoft-AI-Diffusion-Report-2025-H2.pdf}{link}
\item \_job.json --- \href{./archive/\_job.json}{\texttt{./archive/\_job.json}}
\item 20260112\_ms-ai-diffusion-index.md --- \href{./archive/20260112\_ms-ai-diffusion-index.md}{\texttt{./archive/20260112\_ms-ai-diffusion-index.md}}
\end{enumerate}
\section*{Miscellaneous}
\small
\begin{itemize}
\item Generated at: 2026-01-14 23:40:53
\item Duration: 00:09:45 (585.77s)
\item Model: gpt-5.2
\item Quality strategy: none
\item Quality iterations: 0
\item Template: annual\_review
\item Output format: tex
\item PDF compile: enabled
\item Run overview: ./report/run\_overview.md
\item Archive index: ./archive/20260112\_ms-ai-diffusion-index.md
\item Instruction file: ./instruction/20260112\_ms-ai-diffusion.txt
\end{itemize}
\normalsize
\end{document}