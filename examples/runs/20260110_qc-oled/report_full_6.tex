\documentclass[11pt]{article}
\usepackage{kotex}
\usepackage[margin=1in]{geometry}
\usepackage{hyperref}
\usepackage{amsmath,amssymb}
\usepackage{graphicx}
\usepackage{booktabs}
\usepackage{enumitem}
\title{ Federlicht Report - 20260110\_qc-oled }
\author{ Hyun-Jung Kim / AI Governance Team }
\date{ 2026-01-15 }
\begin{document}
\maketitle

\noindent\textit{Federlicht assisted and prompted by "Hyun-Jung Kim / AI Governance Team" — 2026-01-15 05:41}

\section{Abstract}
본 리뷰는 ``지난 12개월(2026-01-10 기준 최근 365일)'' 범위로 수집된 제한된 문헌/웹 근거를 바탕으로, 양자컴퓨팅(quantum computing, QC)이 재료 탐색/설계에 활용되는 기술적 흐름을 OLED 발광재료 개발 관점에서 정리한다. 수집 결과, 동료심사 1차 문헌에서 ``OLED 발광재료에 QC를 직접 적용''한 사례는 본 run에서 충분히 확보되지 않았고(핵심으로 널리 인용되는 npj Computational Materials 2021 논문은 범위 밖), 2025--2026년에 해당하는 자료는 주로 (i) 재료/분자 탐색의 ML 기반 생성(예: GNN 입력 최적화) \href{./archive/openalex/text/W4410193211.txt}{[1]}, (ii) ``양자-AI'' 일반론 또는 로드맵 성격의 리뷰(신뢰도는 재검토 필요) \href{./archive/openalex/text/W4417018335.txt}{[2]}, (iii) 산업/공식 웹 자료(IBM Research 블로그, Mitsubishi Chemical PDF 등)로 나타났다 \href{./archive/tavily\_search.jsonl}{[3]}. 따라서 본고의 핵심 결론은 (1) OLED 관점에서 QC 활용은 여전히 ``여기상태(특히 TADF의 $\Delta E_{ST}$) 예측''이라는 좁은 과업에서 하이브리드(고전-양자) 워크플로로 시연되는 단계이며 \href{./archive/tavily\_search.jsonl}{[3]}, (2) 최근 12개월의 공개 동향은 QC 자체보다는 고전 DFT/ML 파이프라인 고도화가 지배적이고, QC는 오류완화/활성공간 선택/회로자원 제약이 병목으로 남아 있다는 점이다 \href{./archive/openalex/text/W4410193211.txt}{[1]}, \href{./archive/tavily\_search.jsonl}{[3]}. (3) 삼성디스플레이/LG디스플레이/UDC의 QC 적용은 공개자료만으로는 확인 곤란하며, 공개된 내용은 QD-OLED 기술 소개나 공급망/특허 동향에 집중되어 있다 \href{./archive/tavily\_search.jsonl}{[3]}. 본 보고서는 이러한 ``근거 공백''을 명시한 뒤, OLED 발광재료 R\&D에 현실적으로 접목 가능한 2--3개 시나리오(근거 강도 포함)와 12--24개월 전망/의사결정자 질문을 제시한다.  

\section{Introduction}
\subsection{범위, 수집 메타, 한계}
본 run은 2026-01-10 기준 최근 365일 범위에서 Tavily 검색(쿼리당 최대 5개)과 OpenAlex(쿼리당 최대 5개, PDF 다운로드 포함)를 수행하였다 \href{./archive/20260110\_qc-oled-index.md}{[4]}. 그러나 OpenAlex 기반 PDF 다운로드 과정에서 Wiley/ASME/MDPI 등 다수 항목이 403 Forbidden으로 실패하여, 최근(2025--2026) OLED 또는 디스플레이 관련 논문 일부는 메타만 존재하고 본문 검증이 불가능하다 \href{./archive/\_log.txt}{[5]}. 또한 Tavily 결과는 URL과 요약/발췌를 제공하지만, run 폴더에 원문 HTML/PDF가 별도로 아카이빙되지 않은 항목이 다수이므로, 본문에서는 Tavily 발췌를 ``web-derived support''로 취급하고 동료심사 1차 근거로 과도하게 해석하지 않는다 \href{./archive/tavily\_search.jsonl}{[3]}.

\subsection{문제 설정: OLED 발광재료에서 QC의 역할}
OLED 발광재료 개발에서 계산 과업은 (i) 기저/여기 전자구조, (ii) 방사/비방사 전이 및 스핀-궤도 결합, (iii) 열/전하/광 안정성, (iv) 소자 스택 내 상호작용(호스트-도판트, 집합체, 미세구조)로 층위화된다. QC가 직접적으로 노리는 고전 계산의 병목은 주로 전자 상관(correlation)과 다전자 여기상태의 정밀 예측이며, 특히 TADF의 핵심 지표인 단일항-삼중항 에너지 차 $\Delta E_{ST}$, 고에너지 singlet/triplet 상태의 신뢰성 있는 계산이 반복적으로 언급된다(다만 본 run에서 동료심사 1차 근거는 2021년 npj 논문으로 범위 밖이므로, 여기서는 ``웹 발췌'' 수준에서만 언급) \href{./archive/tavily\_search.jsonl}{[3]}.  

\subsection{표기와 약어}
단일항/삼중항 여기상태 에너지를 각각 $E(S_1)$, $E(T_1)$로 두고, $\Delta E_{ST}=E(S_1)-E(T_1)$로 둔다. 변분 양자 고유해법(variational quantum eigensolver, VQE) 기반의 여기상태 접근으로 qEOM-VQE(quantum equation-of-motion VQE), VQD(variational quantum deflation)를 표기한다(해당 약어 자체는 웹 근거에서 반복 등장) \href{./archive/tavily\_search.jsonl}{[3]}.  

\section{Theory & Foundations}
\subsection{양자화학 계산의 표준 형식과 QC 매핑}
분자 전자구조의 출발점은 2차 양자화(second quantization)된 전자 해밀토니안
\begin{equation}
\hat{H}=\sum_{pq} h_{pq} a_p^\dagger a_q+\frac{1}{2}\sum_{pqrs} g_{pqrs} a_p^\dagger a_q^\dagger a_r a_s,
\end{equation}
이며, QC 기반 알고리즘은 (1) 활성공간(active space)에서의 오비탈 절단, (2) 페르미온-큐비트 사상(Jordan--Wigner, Bravyi--Kitaev 등), (3) 파라미터화된 회로(ansatz)로의 변분 최적화를 결합한다. 본 run의 직접적 수학 전개를 제공하는 1차 문헌은 제한적이지만, 산업/학계 웹 발췌에서 ``HOMO/LUMO 활성공간''과 qEOM-VQE/VQD의 적용이 반복적으로 언급되어(특히 TADF 분자의 여기상태) 실제 적용에서는 매우 작은 활성공간에서 시작하는 경향을 시사한다 \href{./archive/tavily\_search.jsonl}{[3]}.  
\paragraph{해석/의미} OLED 발광재료는 다원소/대형 $\pi$-공액 구조가 흔하므로, 활성공간 축소는 필수이나 동시에 물리적 타당성(예: CT 성분, 다중 여기)의 손실 위험을 내포한다. 근거 강도: 중간(웹 발췌 중심).

\subsection{VQE 및 여기상태 변분 방법의 병목}
VQE는
\begin{equation}
E(\boldsymbol{\theta})=\langle \psi(\boldsymbol{\theta})|\hat{H}|\psi(\boldsymbol{\theta})\rangle
\end{equation}
를 고전 최적화로 최소화한다. NISQ에서의 실제 성능은 (i) 회로 깊이와 노이즈, (ii) 측정 샷 수, (iii) barren plateau에 의해 제약된다. 본 run의 ``Quantum-AI Synergy'' 리뷰는 NISQ/FTQC 구분 및 barren plateau, 오류정정 필요성을 개관하지만, 특정 OLED 과업의 재현 가능한 자원 추정/실험 프로토콜은 제공하지 못한다 \href{./archive/openalex/text/W4417018335.txt}{[2]}.  
\paragraph{한계/불확실성} 해당 저널의 학술적 신뢰도(동료심사/인용/데이터)와 일부 정량 주장(예: 특정 기업 성과 수치)의 1차 근거가 본문 발췌에서 확인되지 않는다 \href{./archive/openalex/text/W4417018335.txt}{[2]}. 근거 강도: 낮음(일반론/검증 곤란).

\subsection{QC 외부의 핵심 대안: ML 기반 역설계(inverse design)}
최근 12개월 범위에서 확보된 가장 강한 1차 학술 근거는, GNN 물성 예측기를 미분가능 모델로 간주하고 입력(분자 그래프)을 gradient ascent로 직접 최적화하여 목표 물성(예: HOMO--LUMO gap)을 만족하는 분자를 생성하는 접근이다 \href{./archive/openalex/text/W4410193211.txt}{[1]}. 이 작업은 ``추가적인 생성모델 재학습 없이'' 프록시 예측기만으로 역설계를 수행하고, DFT로 생성 분자의 갭을 재검증하며, 생성 분자에서 프록시 오차가 커짐을 정량적으로 경고한다(예: test set MAE 0.12 eV 대비 생성 분자에서 약 0.8 eV 수준으로 악화) \href{./archive/openalex/text/W4410193211.txt}{[1]}.  
\paragraph{해석/의미} OLED 발광재료에서 QC가 당장 대규모 정확도 우위를 보이기 어려운 상황에서, DFT/ML 기반의 ``역설계+DFT 검증'' 루프가 사실상의 주류 파이프라인임을 시사한다. QC는 단기적으로 이 루프의 특정 병목(여기상태/상관)만 보강하는 형태가 현실적이다. 근거 강도: 높음(동료심사 1차 + DFT 검증 명시).

\section{Methods & Experimental Evidence}
\subsection{본 run에서의 증거 계층화}
\paragraph{주출처(Primary)} OpenAlex로 확보된 동료심사 PDF/텍스트(총 7편)가 주출처이다 \href{./archive/20260110\_qc-oled-index.md}{[4]}. 이 중 OLED\,$\times$\,QC를 직접 다루는 1차 논문은 포함되지 않았고, OLED 관련은 주로 ML/나노기술/일반 리뷰의 주변 언급에 그친다 \href{./archive/openalex/text/W4410193211.txt}{[1]}, \href{./archive/openalex/text/W4406330631.txt}{[6]}.  
\paragraph{지원근거(Supporting; web-derived)} IBM Research 블로그 및 Mitsubishi Chemical PDF, OLED-Info 기사, 기업 보도자료, arXiv/LinkedIn 등이 Tavily로 수집되어 OLED\,$\times$\,QC 연결고리를 제공한다 \href{./archive/tavily\_search.jsonl}{[3]}. 다만 원문 아카이빙 부재 및 2차 매체/유료벽/소셜 포스트 등의 한계를 갖는다.

\subsection{역설계(ML) 파이프라인의 실험적 구성요소}
Nature Communications 2025 논문은 (i) QM9 기반 GNN 프록시 학습, (ii) 그래프 인코딩(인접행렬 $A$와 특성행렬 $F$), (iii) 구조적 제약(대칭성, valence=4 페널티, ``sloped rounding''으로 정수성-미분가능성 절충), (iv) 목표 물성 도달 시 생성 중지, (v) DFT 재검증 및 다양성 지표(Tanimoto distance) 평가를 명시한다 \href{./archive/openalex/text/W4410193211.txt}{[1]}. 특히 ``생성 분자에서 프록시 일반화 성능이 악화''된다는 결과는, 산업 적용 시 반드시 (a) out-of-distribution 검증, (b) 고정밀 계산/실험의 후단 게이트키핑이 필요함을 의미한다.  
\paragraph{근거 강도} 높음.  
\paragraph{OLED 연결의 한계} 해당 논문은 OLED(특히 blue OLED) 관심을 언급하지만, QC 기반 계산이나 실제 OLED 소자 실증은 제공하지 않는다 \href{./archive/openalex/text/W4410193211.txt}{[1]}.

\subsection{OLED 여기상태 QC 적용 주장(지원근거)과 검증 가능성}
IBM Research 블로그는 Mitsubishi Chemical/Keio/JSR 협업을 언급하며, ``noisy quantum computers''에서 오류완화와 ``novel quantum algorithms''로 PSPCz(phenylsulfonyl-carbazole) TADF 후보의 여기상태 전이를 계산한다고 서술한다 \href{./archive/tavily\_search.jsonl}{[3]}. Mitsubishi Chemical 공개 PDF는 qEOM-VQE와 VQD를 사용해 TADF emitter의 여기상태 에너지 예측을 목표로 하며, ``commercial materials''에 대한 excited-state 계산에 QC를 적용한 ``world-first''라는 강한 표현을 포함한다 \href{./archive/tavily\_search.jsonl}{[3]}.  
\paragraph{한계/불확실성} (1) 위 두 자료는 2021년 npj 논문(범위 밖)을 근거로 연결되는 것으로 보이나, 본 run에 해당 npj 원문 PDF/텍스트가 확보되어 있지 않아(링크만 존재) 활성공간, 회로 깊이, 샷 수, 오류완화 프로토콜 등 재현성 핵심 정보를 본 보고서에서 검증할 수 없다 \href{./archive/tavily\_search.jsonl}{[3]}. (2) 따라서 ``OLED에 QC가 유효했다''는 결론은 본 run 근거만으로는 제한적으로만 서술 가능하다.  
\paragraph{근거 강도} 중간(기관/기업 공식 자료의 서술이 있으나, 본 run 내 1차 실험 데이터 미확보).

\subsection{수집 편향(403 다운로드 실패)의 방법론적 영향}
Wiley/ASME/MDPI 등에서 403으로 PDF 확보가 실패하였고, 그 결과 삼성/LG 관련 OLED 논문(예: Advanced Science 2025 blue OLED 성과를 다루는 DOI가 메타로는 존재)도 본문 확인 불가 상태이다 \href{./archive/\_log.txt}{[5]}, \href{./archive/openalex/works.jsonl}{[7]}. 이는 ``지난 12개월'' 범위에서 OLED 발광재료와 계산/알고리즘 진전을 직접 연결하는 1차 근거가 체계적으로 누락될 위험을 의미한다.  
\paragraph{해석/의미} 본 리뷰는 결과적으로 ``QC 적용''보다 ``ML/고전 계산 파이프라인'' 쪽으로 무게가 실릴 수밖에 없으며, 이는 실제 연구동향이라기보다 접근권한/다운로드 실패에 의한 관측 편향일 수 있다. 근거 강도: 높음(로그에 명시) \href{./archive/\_log.txt}{[5]}.

\section{Applications & Benchmarks}
\subsection{OLED 발광재료 유형별 계산 타깃과 QC 적합도}
\subsubsection{형광(Fluorescence) 및 고색순도 청색}
형광 OLED에서 주요 타깃은 큰 발광 에너지(청색), 높은 PLQY, 억제된 비방사 경로, 분자/고분자 안정성이다. 본 run의 1차 학술 근거는 HOMO--LUMO gap 목표 생성이 ``blue OLED 재료 탐색 관심''과 연결될 수 있음을 직접 언급한다 \href{./archive/openalex/text/W4410193211.txt}{[1]}.  
\paragraph{주장} 최근 12개월 범위에서 공개된 강한 방법론적 진전은 QC가 아니라 ``프록시+입력최적화'' 형태의 ML 역설계이다.  
\paragraph{근거} \href{./archive/openalex/text/W4410193211.txt}{[1]}.  
\paragraph{한계} 발광 스펙트럼/여기상태 다중성, 환경(고체, 호스트) 효과는 HOMO--LUMO gap 하나로 포착되지 않는다.  
\paragraph{근거 강도} 높음(방법론), OLED 성능으로의 직접 외삽은 중간 이하.

\subsubsection{인광(Phosphorescence) 및 중금속 복합체}
인광은 SOC가 커서 삼중항 수확이 가능하나, 금속 중심 복합체의 전자상관/상대론 효과, 다중준위 동역학이 계산 병목이다. 본 run에서는 iQCC를 OLED 인광체(Ir(III), Pt(II)) 벤치마크에 적용했다는 arXiv/LinkedIn 발췌가 존재하나, 이는 web-derived support이며 동료심사 출판/데이터 재현성 확인이 불충분하다 \href{./archive/tavily\_search.jsonl}{[3]}.  
\paragraph{근거 강도} 낮음--중간(프리프린트/소셜 발췌 중심).

\subsubsection{TADF 및 CP-OLED(원형편광 OLED)}
TADF의 핵심은 $\Delta E_{ST}$를 작게 하여 RISC를 촉진하는 것이며, 고에너지 singlet/triplet 상태 예측이 중요하다는 점이 웹 발췌에서 반복된다 \href{./archive/tavily\_search.jsonl}{[3]}. QC 적용의 가장 구체적 사례도 PSPCz TADF emitter의 excited states를 qEOM-VQE/VQD로 계산했다는 체인(IBM 블로그--Mitsubishi PDF--npj 링크)이다 \href{./archive/tavily\_search.jsonl}{[3]}.  
\paragraph{주장} OLED\,$\times$\,QC의 가장 현실적인 단기 적용점은 ``TADF 여기상태(특히 $\Delta E_{ST}$)''의 부분공간(활성공간) 정밀화이며, 오류완화가 필수 구성요소로 취급된다.  
\paragraph{근거} \href{./archive/tavily\_search.jsonl}{[3]}.  
\paragraph{한계} 본 run 범위 내에서 동료심사 1차 데이터가 확보되지 않아, 실제 정확도/자원(큐비트 수, 샷 수, 회로 깊이) 및 고전 대안(고정밀 다체 방법) 대비 우위를 평가할 수 없다.  
\paragraph{근거 강도} 중간.

\subsection{산업계(삼성디스플레이, LG디스플레이, UDC)의 공개 정보와 한계}
\subsubsection{Samsung Display}
수집된 삼성 관련 결과는 주로 QD-OLED 구조/반사저감 등의 기술 설명 또는 quantum dot 소재/필름에 관한 홍보성 기사이며, ``양자컴퓨팅을 재료 설계에 사용''했다는 직접 근거는 확인되지 않는다 \href{./archive/tavily\_search.jsonl}{[3]}.  
\paragraph{해석/의미} 공개자료 수준에서 ``quantum''은 quantum dot(양자점) 문맥이 대부분이며, quantum computing과의 혼동 위험이 크다. 근거 강도: 높음(발췌 내용이 명확) \href{./archive/tavily\_search.jsonl}{[3]}.

\subsubsection{LG Display / LG Chem}
LG 관련 공개 보도자료는 p-Dopant 공동개발(공급망/자립화)을 강조하며, QC 적용은 언급하지 않는다 \href{./archive/tavily\_search.jsonl}{[3]}. 또한 Tavily 결과 중 일부는 AI/ML 기반 OLED 설계 리뷰/기사로 연결되지만, 이는 QC가 아니라 고전 양자화학(DFT 등)+ML 파이프라인의 확장이다 \href{./archive/tavily\_search.jsonl}{[3]}.  
\paragraph{근거 강도} 높음(보도자료의 범위), QC 적용 부재는 ``확인 불가''로만 기술 가능.

\subsubsection{Universal Display Corporation (UDC)}
UDC 관련은 특허자산 인수 및 공급/라이선스 계약 등 사업/지식재산 동향이 주이며 QC 언급은 없다 \href{./archive/tavily\_search.jsonl}{[3]}.  
\paragraph{해석/의미} OLED 재료 기업의 공개 활동은 IP/공급계약 중심이며, 계산기술(QC 포함)의 내부 활용은 비공개로 남는 경향이 있다. 근거 강도: 중간(부재의 증명은 불가하므로, ``공개자료에 없음''으로 제한).

\subsection{대표 시나리오 3종(주장--근거--한계--의미)}
\subsubsection{시나리오 A: ``ML 역설계 + DFT 검증''이 주도, QC는 후단 정밀화로 제한}
\paragraph{주장} 단기(12--24개월)에는 OLED 재료 탐색의 주류가 ML 기반 역설계(프록시 모델 입력최적화, 생성모델)이며, QC는 특정 소수 후보에서의 여기상태 정밀화/검증 모듈로만 투입될 가능성이 높다.  
\paragraph{근거} GNN 예측기를 분자 생성기로 역이용하고 DFT로 검증하는 파이프라인이 명시되어 있으며, 프록시 일반화 실패를 정량 경고한다 \href{./archive/openalex/text/W4410193211.txt}{[1]}.  
\paragraph{한계} DFT 검증 비용이 여전히 크고, 소자-수명/공정 적합성 데이터가 결합되지 않으면 산업 효용이 제한된다.  
\paragraph{의미} QC 투입의 KPI는 ``전주기 속도''가 아니라 ``후단 고난도 물성(여기상태)에서 고전 방법 대비 비용-정확도 우위''로 재정의되어야 한다.  
\paragraph{근거 강도} 높음(ML 파이프라인), QC 결합은 해석적 제안으로 근거 강도 중간.

\subsubsection{시나리오 B: ``TADF 여기상태 QC PoC''의 산업-학계 협업 지속(오류완화 중심)}
\paragraph{주장} 기업-학계 협업은 TADF emitter의 excited states를 대상으로 qEOM-VQE/VQD 및 오류완화 기법을 적용하는 PoC를 지속하며, ``활성공간을 작게 유지한 채 신뢰성 있는 $\Delta E_{ST}$''를 1차 목표로 삼는다.  
\paragraph{근거} IBM Research 블로그 및 Mitsubishi Chemical PDF에서 PSPCz TADF와 qEOM-VQE/VQD, 오류완화가 핵심 키워드로 등장한다 \href{./archive/tavily\_search.jsonl}{[3]}.  
\paragraph{한계} 본 run 내 동료심사 1차 데이터 미확보로 인해 성능 재현/비교가 불가하고, 범용화(다양한 scaffold, 고체환경)로의 확장도 불명확하다 \href{./archive/tavily\_search.jsonl}{[3]}.  
\paragraph{의미} 산업 적용에서 가장 먼저 표준화되어야 할 것은 ``오류완화 포함 프로토콜(샷/회로/활성공간/기준오차)''과 ``고전 기준선(benchmark)''이다.  
\paragraph{근거 강도} 중간.

\subsubsection{시나리오 C: 디스플레이 대기업의 QC 활용은 비공개(간접 신호만 존재)로 남고, 공개는 QD-OLED/소재 자립화 중심}
\paragraph{주장} 삼성/ LG/ UDC의 공개자료에서는 QC 기반 재료탐색의 직접 증거가 거의 없으며, 공개 커뮤니케이션은 QD-OLED 공정/소재, 공급망, 특허에 집중된다.  
\paragraph{근거} Samsung Display 기술 페이지 및 삼성 뉴스룸 콘텐츠는 quantum dot/필름/반사저감 등이며 QC 언급이 아니다 \href{./archive/tavily\_search.jsonl}{[3]}. LG는 p-Dopant 공동개발 보도자료가 대표적이다 \href{./archive/tavily\_search.jsonl}{[3]}. UDC는 특허자산 인수/공급계약 발표가 중심이다 \href{./archive/tavily\_search.jsonl}{[3]}.  
\paragraph{한계} ``공개정보 부재''는 내부 활동 부재를 의미하지 않는다. 다만 본 run 범위에서 확인 가능한 것은 ``부재''뿐이다.  
\paragraph{의미} 의사결정자는 (i) 내부 PoC의 기술-사업 KPI 정의, (ii) 외부 공개 범위(특허/논문화/협업 발표) 전략을 분리해 설계해야 한다.  
\paragraph{근거 강도} 중간(부재 기반 결론의 구조적 한계 존재).

\section{Synthesis & Outlook}
\subsection{학계-산업 간 간극: 가장 큰 병목}
\paragraph{(1) 재현성/표준 벤치마크의 결핍} OLED\,$\times$\,QC 적용에서 필요한 것은 특정 분자 사례의 성공담이 아니라, 활성공간 선정, ansatz, 오류완화, 샷 예산, 고전 기준선(예: 고정밀 여기상태 방법)과의 비교가 포함된 공개 벤치마크이다. 현재 본 run에서 이는 주로 웹 발췌로만 관측된다 \href{./archive/tavily\_search.jsonl}{[3]}.  
\paragraph{(2) 목표함수의 불일치} 산업은 EQE/수명/수율/공정호환성을 요구하지만, 계산(특히 QC)은 제한된 분자 단위 전자구조 지표(예: $\Delta E_{ST}$, 에너지 갭)에 머물기 쉽다. ML 역설계 논문도 갭 목표 생성은 가능하나, 생성 분자에서 프록시 오차가 급증함을 보여 ``목표함수-현실지표'' 간극을 드러낸다 \href{./archive/openalex/text/W4410193211.txt}{[1]}.  
\paragraph{(3) 데이터 파이프라인의 폐쇄성} 산업 실험 데이터(소자 스택, 공정 조건, 열화 데이터)는 핵심 자산이어서 공개가 어렵다. 이때 공개 가능한 형태의 ``요약 통계/프로토콜''조차 없으면 학계 알고리즘의 산업 검증이 지연된다. 본 run에서 대기업의 QC 적용이 공개되지 않는 점은 이 문제를 간접적으로 시사한다 \href{./archive/tavily\_search.jsonl}{[3]}.

\subsection{향후 12--24개월 전망: ``QC 단독 혁신''이 아니라 ``하이브리드 모듈화'' 경쟁}
\paragraph{주장} 향후 12--24개월 동안 공개 영역에서 관측될 가능성이 높은 변화는, (i) ML 역설계의 산업 파이프라인화(프록시 모델의 OOD 관리 및 후단 검증 강화), (ii) QC의 역할을 ``여기상태/다체상관'' 등 좁은 하위과업의 검증 모듈로 위치시키는 하이브리드 워크플로 정착, (iii) 오류완화/회로 압축/활성공간 선택 등 ``계산 프로토콜 표준'' 경쟁이다.  
\paragraph{근거} ML 역설계의 구체적 파이프라인(입력최적화+DFT 검증+OOD 경고)이 동료심사 1차로 제시된다 \href{./archive/openalex/text/W4410193211.txt}{[1]}. OLED\,$\times$\,QC는 공식/산업 문건에서 qEOM-VQE/VQD와 오류완화 중심의 주장 형태로 제시된다 \href{./archive/tavily\_search.jsonl}{[3]}.  
\paragraph{한계/불확실성} 본 run에서는 OLED\,$\times$\,QC의 동료심사 1차 재현 데이터가 확보되지 않았고, 또한 Wiley/ASME/MDPI 다운로드 실패(403)가 최근 OLED 관련 1차 근거를 누락시켰을 가능성이 크다 \href{./archive/\_log.txt}{[5]}. 따라서 ``QC가 실제로 산업 성능지표를 개선할 것''이라는 예측은 강하게 단정할 수 없다.  
\paragraph{해석/의미} 의사결정의 초점은 ``QC가 언제 DFT를 대체하는가''가 아니라, ``(A) 현재의 ML/DFT 파이프라인에서 가장 비싼/불확실한 물성 추정 구간이 무엇이며, (B) 그 구간을 QC(또는 고전 고정밀 방법)가 모듈로써 얼마나 줄일 수 있는가''로 이동해야 한다. 근거 강도: 중간(주로 파이프라인 관측 및 논리적 귀결).

\subsection{의사결정자용 후속 질문(공개자료 기반으로 즉시 설정 가능한 체크리스트)}
\paragraph{주장} OLED 발광재료 개발 조직이 QC 도입 여부/우선순위를 판단하려면, 기술성숙도보다 먼저 ``문제정의(benchmark)''와 ``데이터/검증 게이트''를 명문화해야 한다.  
\paragraph{근거} 본 run에서 QC 적용은 재현 가능한 실험 파라미터가 결여된 서술(웹 기반)이 많고 \href{./archive/tavily\_search.jsonl}{[3]}, 반대로 ML 역설계는 파이프라인과 오차 붕괴(OOD)를 정량으로 드러내어 ``검증 게이트''의 필요를 명확히 한다 \href{./archive/openalex/text/W4410193211.txt}{[1]}.  
\paragraph{한계} 아래 질문은 본 run 자료의 한계(1차 데이터 부재, 403 편향)를 전제로 한 ``실무적 프레임''이며, 특정 기업의 실제 내부 전략을 추론하지 않는다 \href{./archive/\_log.txt}{[5]}, \href{./archive/tavily\_search.jsonl}{[3]}.  
\paragraph{해석/의미(질문 목록)}  
(1) 목표 과업을 $\Delta E_{ST}$, SOC, 다중 여기상태 순서로 분해했을 때, 현재 DFT/TDDFT/고전 다체 방법의 오차/비용 프로파일은 무엇인가? (내부 벤치마크 부재 시 외부 공개 벤치마크부터 수립 필요) \href{./archive/tavily\_search.jsonl}{[3]}.  
(2) 후보 분자군에서 OOD가 발생하는 지점을 어떻게 탐지하고(예: 예측 불확실성), 어떤 후단 게이트(DFT/실험)를 둘 것인가? \href{./archive/openalex/text/W4410193211.txt}{[1]}.  
(3) QC PoC를 한다면, 활성공간 규칙(오비탈 선택), 회로 깊이 상한, 샷 예산, 오류완화 기법, 그리고 고전 기준선(동일 활성공간에서의 고전 다체 계산)을 무엇으로 고정할 것인가? \href{./archive/tavily\_search.jsonl}{[3]}.  
(4) 대기업(삼성/LG) 및 재료기업(UDC) 관련 공개자료에서는 QC 근거가 거의 없었는데, 이는 ``비공개''일 수 있다. 내부적으로는 어떤 수준까지 공개/특허/논문화가 가능한가(대외 협업 전략 포함)? \href{./archive/tavily\_search.jsonl}{[3]}.  
근거 강도: 중간(질문 도출은 근거 관측+정합성 요구에 기반).

\section{Appendix}
\subsection{수집 설정(메타) 요약}
\paragraph{주장} 본 리뷰의 결론은 ``최근 365일'' 범위 제한과 수집 옵션에 강하게 의존한다.  
\paragraph{근거} Run date 및 옵션(``days 365'', ``--max-results 5'', ``--download-pdf'', ``--openalex --oa-max-results 5'')이 명시되어 있다 \href{./archive/20260110\_qc-oled-index.md}{[4]}.  
\paragraph{한계/불확실성} 쿼리당 최대 결과 수 제한과 PDF 다운로드 실패는 체계적 누락을 유발할 수 있다 \href{./archive/\_log.txt}{[5]}.  
\paragraph{해석/의미} 본 리뷰는 ``완전한 문헌조사''가 아니라, 재현 가능한 수집 파이프라인의 산출물에 대한 ``근거-기반 스냅샷''으로 해석해야 한다. 근거 강도: 높음.

\subsection{다운로드 실패(403)로 인한 누락 가능성}
\paragraph{주장} OLED 및 디스플레이 관련 2025--2026 논문 일부가 접근 제한(403)으로 본문 검증에서 누락되었다.  
\paragraph{근거} Wiley/ASME/MDPI PDF 다운로드 403 실패가 로그에 반복 기록되어 있다 \href{./archive/\_log.txt}{[5]}.  
\paragraph{한계/불확실성} 누락된 논문 중 OLED\,$\times$\,QC 직접 근거가 포함되었을 가능성을 배제할 수 없다(따라서 본문 결론의 ``QC 근거 공백''은 관측 편향을 포함할 수 있음) \href{./archive/\_log.txt}{[5]}.  
\paragraph{해석/의미} 후속 작업에서는 (i) 기관 구독을 통한 PDF 확보, (ii) 저자 자가아카이브/프리프린트 탐색, (iii) 쿼리 확장 및 결과 수 상향을 통해 근거 공백을 재평가해야 한다. 근거 강도: 높음(실패 사실), 영향 추정은 중간.

\subsection{본 run에서 ``오프토픽/대조''로만 유효한 문헌(주의 표기)}
\paragraph{주장} 일부 OpenAlex 수집 문헌은 ``quantum'' 용어를 포함하더라도 OLED\,$\times$\,QC(양자컴퓨팅 기반 화학/재료탐색)과 주제 적합성이 낮다.  
\paragraph{근거} ``quantum materials'' 리뷰는 양자물질(강상관, 토폴로지 등)의 정의/응용을 다루며 본 보고서의 ``QC로 OLED 발광재료를 설계/탐색''이라는 초점과 직접 일치하지 않는다 \href{./archive/openalex/text/W4406477905.txt}{[8]}. organoluminophores 리뷰는 OLED/TADF 약어를 포함하나 주 내용은 electrospinning/nano-dots 응용이다 \href{./archive/openalex/text/W4406330631.txt}{[6]}.  
\paragraph{한계/불확실성} 이러한 문헌은 배경/용어 대비에는 유용할 수 있으나, ``OLED\,$\times$\,QC 성과''의 직접 근거로 인용할 수는 없다.  
\paragraph{해석/의미} 본문에서는 ``근거 강도''를 낮게 또는 ``배경''으로만 처리하는 것이 타당하다. 근거 강도: 높음(텍스트 내용으로 확인 가능) \href{./archive/openalex/text/W4406477905.txt}{[8]}, \href{./archive/openalex/text/W4406330631.txt}{[6]}.

\section*{Figures}
\paragraph{Figures referenced.} Figure~\ref{fig:1}: Source PDF: W4410193211.pdf Figure~\ref{fig:2}: Source PDF: W4417018335.pdf Figure~\ref{fig:3}: Source PDF: W4417018335.pdf

\begin{figure}[htbp]
\centering
\includegraphics[width=\linewidth]{report\_assets/figures/.\_archive\_openalex\_pdf\_W4410193211.pdf-5f54c461.png}
\caption{Source: \\texttt{./archive/openalex/pdf/W4410193211.pdf}, page 3.}
\label{fig:1}
\end{figure}

\begin{figure}[htbp]
\centering
\includegraphics[width=\linewidth]{report\_assets/figures/.\_archive\_openalex\_pdf\_W4417018335.pdf-0c4c4bec.png}
\caption{Source: \\texttt{./archive/openalex/pdf/W4417018335.pdf}, page 1.}
\label{fig:2}
\end{figure}

\begin{figure}[htbp]
\centering
\includegraphics[width=\linewidth]{report\_assets/figures/.\_archive\_openalex\_pdf\_W4417018335.pdf-3d08f980.png}
\caption{Source: \\texttt{./archive/openalex/pdf/W4417018335.pdf}, page 21.}
\label{fig:3}
\end{figure}

\section*{Report Prompt}
\begin{verbatim}
지난 12개월 동안의 양자컴퓨팅 기반 재료 연구 및 산업 적용 동향을 OLED 발광재료 개발 관점에서 분석해줘.
특히 다음을 포함해 정리해줘:
- 양자컴퓨팅이 재료 탐색/설계에 쓰이는 주요 흐름(알고리즘, 워크플로, 데이터 파이프라인)
- OLED 발광재료(형광/인광/TADF/CP-OLED 등) 탐색에 적용된 접근과 성과
- 삼성디스플레이, LG디스플레이, UDC 등 산업계의 시도와 공개 정보의 한계
- 학계 연구 흐름과 산업 적용 간의 간극, 가장 큰 병목과 해결 과제
- 2~3개의 대표적 연구/산업 시나리오를 근거와 함께 제시
- 향후 12~24개월 내 기대되는 변화와 의사결정자용 후속 질문

출처는 논문/리뷰/공식 발표/신뢰 가능한 업계 자료를 우선으로 하고, 웹 검색으로 보강한 내용은 “supporting”으로 구분해 활용해줘.

작성 스타일/톤:
- 학술/기술 전문가 리뷰 논문 수준의 엄밀한 서술체로 작성해줘. (교수식 서술, 과장/마케팅 톤 금지)
- 각 핵심 주장마다 “주장 → 근거(출처) → 한계/불확실성 → 해석/의미” 흐름으로 구성해줘.
- 근거의 강도를 짧게 표기해줘(예: 근거 강도: 높음/중간/낮음).
- 기술적 메커니즘(알고리즘·실험 조건·데이터/공정 조건)을 가능한 한 명시하고, 재현성/스케일링 가능성도 평가해줘.
- 학계 결과와 산업 적용의 간극(수율, 비용, 수명, 신뢰성, 공정 호환성 등)을 비판적으로 다뤄줘.
- 핵심 용어는 짧게 정의하고, 유사 개념은 비교·대조해줘.
- 인용은 “실제 출처” 기준으로 하며, 추론/의견은 명확히 구분해줘.
\end{verbatim}
\section*{References}
\renewcommand{\labelenumi}{[\arabic{enumi}]}
\begin{enumerate}
\item Using GNN property predictors as molecule generators --- \href{./archive/openalex/text/W4410193211.txt}{\texttt{./archive/openalex/text/W4410193211.txt}} \textit{citations: 8}
\item Quantum-AI Synergy and the Framework for Assessing Quantum Advantage --- \href{./archive/openalex/text/W4417018335.txt}{\texttt{./archive/openalex/text/W4417018335.txt}} \textit{citations: 0}
\item Tavily search index (\href{./archive/tavily\_search.jsonl}{\texttt{./archive/tavily\_search.jsonl}}) --- selected sources:
\begin{itemize}
\item Unlocking today's quantum computers for OLED applications --- \href{https://research.ibm.com/blog/quantum-for-oled}{link}
\item Mitsubishi Chemical, Deloitte Tohmatsu and Classiq manage to ... --- \href{https://www.oled-info.com/mitsubishi-chemcial-deloitte-tohmatsu-and-classiq-manage-dramatically-improve}{link}
\item [PDF] A Joint Paper on Prediction of Optical Properties of OLED Materials ... --- \href{https://www.mcgc.com/english/news\_mcc/2021/\_\_icsFiles/afieldfile/2021/05/26/qhubeng.pdf}{link}
\item Products/Technology – QD-OLED - Samsung Display --- \href{https://www.samsungdisplay.com/eng/tech/quantum-dot.jsp}{link}
\item Transforming Materials Science Through Quantum ... --- \href{https://www.oled-info.com/transforming-materials-science-through-quantum-collaboration}{link}
\item LG Display and LG Chem Jointly Develop Essential OLED Material ... --- \href{https://www.lgcorp.com/media/release/26853}{link}
\end{itemize}
\item 20260110\_qc-oled-index.md --- \href{./archive/20260110\_qc-oled-index.md}{\texttt{./archive/20260110\_qc-oled-index.md}}
\item \_log.txt --- \href{./archive/\_log.txt}{\texttt{./archive/\_log.txt}}
\item Electrospinning vs Fluorescent Organic Nano-Dots: A Comparative Review of Nanotechnologies in Organoluminophores Utilization --- \href{./archive/openalex/text/W4406330631.txt}{\texttt{./archive/openalex/text/W4406330631.txt}} \textit{citations: 2}
\item OpenAlex works index (\href{./archive/openalex/works.jsonl}{\texttt{./archive/openalex/works.jsonl}}) --- selected sources:
\begin{itemize}
\item Quantum-AI Synergy and the Framework for Assessing Quantum Advantage --- \href{https://doi.org/10.63721/25jpair0118}{link} \textit{citations: 0}
\end{itemize}
\item Exploring quantum materials and applications: a review --- \href{./archive/openalex/text/W4406477905.txt}{\texttt{./archive/openalex/text/W4406477905.txt}} \textit{citations: 22}
\end{enumerate}
\section*{Miscellaneous}
\small
\begin{itemize}
\item Generated at: 2026-01-15 05:41:16
\item Duration: 00:13:08 (788.52s)
\item Model: gpt-5.2
\item Quality strategy: none
\item Quality iterations: 0
\item Template: review\_of\_modern\_physics
\item Output format: tex
\item PDF compile: enabled
\item Run overview: ./report/run\_overview.md
\item Archive index: ./archive/20260110\_qc-oled-index.md
\item Instruction file: ./instruction/20260110\_qc-oled.txt
\item Figure candidates: ./report\_views/figures\_preview.html
\end{itemize}
\normalsize
\end{document}