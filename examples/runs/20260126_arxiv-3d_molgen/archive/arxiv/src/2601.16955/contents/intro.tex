\section{Introduction}
\begin{figure}[ht]
\centering 
\includegraphics[width=0.95\columnwidth]{visuals/title_figure.png}
\caption{Molecule $\bm{\mathcal{M}}$ is generated from motifs $\mathcal{M}$ in a joint flow on the product space of rigid frames $\mathbf{T}$ and motif classes $m$.}
\label{title_figure}
\vspace{-3mm}
\end{figure}
The generation of 3D molecular structures is a cornerstone of \textit{in silico} drug discovery and material design. Recent advances in deep learning have enabled the development of powerful generative models that treat molecules as point clouds of atoms, utilising $\mathrm{E(3)}$- and $\mathrm{SE(3)}$-equivariant diffusion-based frameworks to determine atomic coordinates \cite{edm, geoldm}. While these atom-based approaches achieve impressive performance, they operate at a low level of abstraction, discarding the rich chemical modularity inherent to molecular structures. In the realm of molecular graph generation, however, this hierarchical nature is widely exploited by fragment-based methods \cite{jtvae, magnet}, which assemble molecules from chemically meaningful motifs or scaffolds to capture high-level semantics and ensure validity. In the domain of 3D protein structure generation, such a modular perspective has also proven highly effective. Seminal works such as \textsc{AlphaFold2} \cite{alphafold2} and \textsc{FrameDiff} \cite{se3diffusion} demonstrated the efficacy of abstracting amino acid residues as \textit{rigid frames} in $\mathrm{SE(3)}$, decoupling backbone geometry from local side-chain details. However, lifting these ideas to general drug-like molecule generation in 3D presents significant challenges: unlike proteins, which are linear chains of a fixed set of residues, small molecules exhibit arbitrary branching, diverse topologies, and a vastly larger vocabulary of chemical motifs. 

In this work, we propose to bridge this gap by treating general molecules as collections of \textit{rigid motifs}. By decomposing molecules into chemically meaningful rigid fragments, we lift the generative task from the $\mathbb{R}^3$ atom space to the $\mathrm{SE(3)}$ manifold of fragments. We adopt a multimodal flow matching framework that jointly learns the discrete distribution of motif types and their continuous spatial configuration, illustrated in Figure \ref{title_figure}. This formulation enables us to generate high-fidelity molecular structures with significantly fewer sampling steps, leveraging chemically inspired representations that are substantially more concise than those of standard all-atom approaches.
\vspace{-2mm} 
\paragraph{Contributions} Our primary contributions are as follows. We propose \oursacro, a framework for 3D molecule generation that parametrises molecules as sets of rigid-body motifs in $\mathrm{SE}(3)$ rather than individual atoms. We develop a data-driven fragmentation and canonicalisation strategy that handles arbitrary molecular topologies and accounts for motif symmetries.
Further, we adapt the multimodal flow \citep{discrete_flows} to the task of \textit{de novo} 3D structure generation of drug-like molecules. This formulation allows us to natively handle the joint generation of discrete fragment identities and their continuous geometric configurations without relying on autoregressive steps or learned decoding back onto the atom-level structure. Finally, we empirically demonstrate that our method achieves superior performance compared to state-of-the-art atom-based baselines on medium- and large-sized molecules in the \textsc{GEOM-Drugs} benchmark \cite{geom} and scales to the larger molecules of the \textsc{QMugs} dataset \cite{qmugs}. \oursacro produces high-quality molecular structures using an order of magnitude fewer generation steps and excels at conditional generation tasks.