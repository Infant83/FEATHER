\vspace{-4mm} 
\section{Conclusion}
\vspace{-1mm} 
In this work, we introduced \oursacro, a novel generative framework for 3D molecules that operates on \textit{rigid motifs} rather than individual atoms. By combining the rigid-motif decomposition strategy with the multimodal flow matching objective, we generate drug-like molecules via flows on the $\mathrm{SE(3)}$ manifold coupled with discrete categorical flows. Our empirical evaluation on standard benchmarks demonstrates that this higher-level representation yields improved stability on larger molecules compared to established all-atom methods, while offering more concise molecular representations as well as significant advantages in sampling efficiency and conditional generation capabilities.
\vspace{-3mm} 
\paragraph{Limitations and Future Work} This method, while extending \textit{motif-based} molecular graph generation to the 3D space, also inherits its limitations. Most notably, the reliance on a predefined vocabulary introduces a trade-off between vocabulary size and generalisation: as the vocabulary grows to capture larger motifs, the frequency of individual classes drops, leading to lower coverage of uncommon substructures during sampling. An alternative, \textit{scaffold-based} generation \cite{magnet}, however, is non-trivial in 3D due to the vastly different spatial geometries of fragments of the same shape, and thus its integration into our method constitutes an exciting direction for future work. Of potential interest could also be an extension of our method to models that jointly generate 3D and 2D molecular structures (e.g., \citealp{codesign}), thereby bridging motif-based generation with explicit modelling of all bonds.
\vspace{-3mm} 
\paragraph{Broader Impact} Generative models for molecules have the potential to accelerate \textit{in-silico} discovery
and the design of novel drugs and materials, but they also carry potential dangers, as such models could be misused for designing chemicals with socially adverse properties.