\vspace{-2mm} 
\section{Related Work}
\label{Related Work}
\vspace{-1mm} 
\paragraph{Fragment-Based Molecule Generation} In the task of molecular graph generation, several approaches to tokenising molecules into motifs exist, which can be classified into chemically inspired and data-driven ones. The former group \cite{hiervae, moler, jtvae, frag_fm} focuses on top-down separation of acyclic and cyclic parts in the molecule, while the latter group \cite{psvae, micam} adopts the bottom-up merging strategy starting from individual atoms. As an alternative to motif-based methods, \citet{magnet} proposed \textit{scaffold-based} fragmentation, in which the molecular assembly starts from basic geometric shapes and attributes the chemical properties in the subsequent stages of generation. To the best of our knowledge, neither motif-based nor scaffold-based fragmentation has been widely explored in 3D. A notable exception is \textsc{HierDiff} \cite{hierdiff}, which utilises a hierarchical diffusion framework to position coarse fragment nodes in 3D space. In contrast to our approach, which generates the full $\mathrm{SE}(3)$ configuration of rigid motifs without auxiliary mechanisms, \textsc{HierDiff} relies on a learnable decoding procedure to resolve the specific identities and geometries of the fragments, thereby taking a fundamentally different route.
\vspace{-3mm} 
\paragraph{3D Molecule Generation} Generation of spatial molecular structure is primarily tackled with flow-based \cite{enflows, enlinker, flowmol} and autoregressive \cite{autoreg1, autoreg2, quetzal} models on atom coordinates. A notable exception to these directions is the work of \citet{voxel}, where 3D molecules are represented as atomic densities on regular grids. Existing approaches to 3D molecular generation can also be roughly classified into two categories based on how they treat bonds. The seminal work of \citet{edm} uses a lookup table to infer bond types from pairwise atom distances, which is the approach we adopt. Recently, a line of work \cite{codesign, midi, semlaflow, moldiff} has proposed a joint modelling of the 2D molecular graph topology and 3D atom coordinates. While they show improvement over point cloud approaches, we perform our experiments without modelling the graph structure beyond intra-motif connectivity, i.e., beyond the bond structure within the fragments; therefore, we do not directly compare our method to these approaches.
% Recently, there has been an emerging interest in the \textit{trans-dimensional} generation of molecules \cite{branchingflows, jumpdiffusion}, i.e., where molecules are generated atom-by-atom in a single reverse process, rather than residing on a fixed, predefined support of $N$ atoms, which can be seen as a medium between autoregressive and flow-based approaches. We note that, while our number of fragments $K$ is fixed for a given molecule during generation, as in any conventional flow matching, the number of atoms $\hat{N}(K)$ generated can be vastly different due to the difference in rigid motifs' sizes. From that perspective, our framework offers flexibility, albeit limited, in the number of atoms generated, without relying on branching or jump process ideas.
\vspace{-3mm} 
\paragraph{Rigid-body Generation} Parametrising protein residues as rigid frames has been originally introduced in the seminal work of \citet{alphafold2} and received a widespread adoption in the subsequent methods for protein structure prediction and design \cite{rfdiffusion, se3diffusion, foldflow}. Generative modelling with rigid frames outside the protein application, however, is limited. A concurrent work \cite{sigmadock} explores rigid-fragment $\mathrm{SE}(3)$ diffusion for molecular docking, which is an orthogonal task to ours. \textit{De novo} generation of 3D structure of general molecules from rigid fragments thus remains underexplored, which is a gap the present paper aims to fill.