\section{Abstract}
본 보고서는 TADF(thermally activated delayed fluorescence) 유기 발광체의 고속 스크리닝과 설계 가속을 목표로 한 2025년 arXiv 연구 2편과, PhOLED(Ir(III))의 효율/수명 예측을 위한 2019년 속도론(kinetic) 프레임 연구 1편을 하나의 “가속 설계 스레드”로 종합한다. 첫 논문은 747개 실험 특성화 TADF 발광체에서 sTDA-xTB 및 sTD-DFT-xTB가 TD-DFT 대비 99\% 이상 비용을 절감하면서 방법 간 내부 일관성(예: $\Delta E_{ST}$ Pearson $r\approx 0.82$)을 제공해 ‘절대 예측’보다 ‘상대 랭킹’에 적합함을 보인다[1]. 실험 $\Delta E_{ST}$ 312개에 대한 평균절대오차(MAE) $\approx 0.17$ eV는 수직 근사(vertical approximation) 한계로 해석된다[1]. 두 번째 논문은 동일 코퍼스를 기반으로 NTO(Natural Transition Orbital)에서 유도한 전하이동(CT) 디스크립터(특히 hole-electron overlap $S_{he}$)를 도입하고, SVR로 $\Delta E_{ST}$ 예측에서 MAE $=0.024$ eV, $R^2=0.960$을 보고하며, 액티브 러닝이 무작위 샘플링 대비 약 25\%의 라벨 요구량 감소를 주장한다[2]. 세 번째 논문은 Ir(III) 복합체에서 효율 $\Phi$ 및 수명 $\tau$를 경쟁 소거 rate의 합으로 표현하고, 온도 의존 비복사 소거 $k_{nr}(T)$를 명시적으로 포함해 상온에서의 효율/수명 저하를 재현 가능한 메커니즘으로 설명한다[3]. 종합적으로, 본 스레드는 (i) 저비용 전자구조 파이프라인의 대규모 검증, (ii) 물리 기반 디스크립터+ML/AL로의 전환, (iii) 최종 성능 지표를 동역학·온도 축으로 확장하는 방향을 제시하지만, 수직 근사·라벨/분할 정합성(누수)·고체상/호스트 효과·온도 동역학의 통합 부재가 핵심 갭으로 남는다.

\section{Executive Summary}
본 원고는 세 편의 arXiv 논문을 하나의 “가속 설계 스레드”로 연결해 기술한다: (i) 747개 문헌·실험 기반 TADF 코퍼스에서 xTB 기반 sTDA/sTD-DFT 파이프라인을 대규모로 검증하여 HTVS(고속 가상 스크리닝)의 속도·재현성을 확보하고[1], (ii) 동일 코퍼스에서 NTO 유래 전하이동(CT) 디스크립터(특히 hole–electron overlap $S_{he}$)를 구축해 설명가능 ML/액티브 러닝으로 탐색 효율을 추가로 끌어올리며[2], (iii) 단일 전자구조 지표($\Delta E_{ST}$ 등)를 넘어 경쟁 소거 rate 및 온도 의존 $k_{nr}(T)$로 효율/수명 목표함수를 재정의하는 속도론 프레임을 제시한다[3].

편집자 관점에서 본 스레드의 핵심 통찰은 “속도(저비용 전자구조)–정확도(디스크립터+ML)–목표함수(효율/수명·온도)”의 단계적 확장이며, 동시에 (a) 수직 근사로 인한 실험 절대 오차, (b) [2]의 라벨 정의 및 분할 단위가 만드는 누수/과대평가 위험, (c) 고체상/호스트 및 온도 동역학을 TADF 워크플로에 통합하지 못한 갭이 주요 리스크로 남는다. 결론에서는 이 리스크를 직접 겨냥하는 5개의 실행 가능한 연구 방향과 다음 단계 질문을 제시한다.

\section{Scope \& Methodology}
본 리뷰는 본 실행(run)에서 제공된 세 편의 arXiv 텍스트를 유일한 근거로 사용한다. 본문 인용은 번호 체계로 통일하며, 매핑은 다음과 같다: [1] arXiv:2511.00922v1, [2] arXiv:2512.06029v1, [3] arXiv:1901.01201v1. 텍스트 추출물의 파일 경로는 내부 증거 저장소이므로 본문 근거 표기에는 사용하지 않는다.

종합 방법은 (i) 각 논문의 목적·가정·프로토콜·정량 결과를 동일 축(라벨 정의, 평가 설계, 근사/모델링 가정, 계산비용, 일반화 범위)으로 재정리하고, (ii) 세 논문이 암묵적으로 공유하거나 충돌하는 설계 선택(수직 근사, 용매/환경 취급, 목표변수 정합성)을 중심으로 통합 리스크/갭을 구조화하며, (iii) 논문이 직접 시사하는 범위 내에서 실행 가능한 후속 연구 방향을 제안하는 것이다.

\section{Introduction}
TADF 및 PhOLED 발광체 설계는 “광물성(효율, 색, 수명)”과 “실험 비용/시간” 사이의 구조적 긴장을 갖는다. TADF의 경우 작은 싱글렛-트립렛 갭 $\Delta E_{ST}$가 열적 업컨버전(RISC)의 전제 조건이며, 이를 위해 HOMO-LUMO 분리(CT 성격)가 요구되지만, 동시에 빠른 RISC를 위해 SOC/oscillator strength 같은 ‘겹침’이 필요해 상충 조건이 발생한다[2]. 고정밀 excited-state 계산은 비용이 높아 대규모 화학공간 탐색에 병목이 되고, 저비용 근사는 체계적 오차를 낳는다[1]. 한편 PhOLED(Ir(III))에서는 경쟁 비복사 채널과 온도 의존 소거가 얽혀 효율 계산이 “elusive”하다는 진단이 제기되어 왔다[3].

본 리뷰의 목표는 세 논문을 (1) HTVS 파이프라인 검증[1], (2) 물리 디스크립터+ML/AL로의 탐색 효율화[2], (3) 경쟁 rate 및 온도 의존 소거를 포함하는 성능 목표함수 확장[3]이라는 단계로 연결해, “무엇을 빠르게 계산할 것인가(전자구조) $\to$ 무엇을 학습할 것인가(디스크립터/라벨) $\to$ 무엇을 최적화할 것인가(효율/수명/온도)”로 확장되는 설계 아젠다를 명확히 하는 데 있다.

연구 질문(Research Questions, RQ)은 다음과 같다.
\subsection*{RQ1} xTB 기반 sTDA/sTD-DFT 파이프라인은 대규모 TADF 데이터에서 어떤 수준의 정확도와 랭킹 신뢰도를 제공하며, 그 한계는 무엇인가[1]?
\subsection*{RQ2} NTO 기반 CT 디스크립터(특히 $S_{he}$)는 $\Delta E_{ST}$ 예측에서 어떤 설명가능성과 성능 개선을 제공하는가[2]?
\subsection*{RQ3} 액티브 러닝은 라벨을 얼마나 절감하며, 어떤 쿼리 전략 비교가 제시되는가[2]?
\subsection*{RQ4} 경쟁 소거 채널과 온도 의존 $k_{nr}(T)$를 포함하는 속도론 프레임은 최종 성능 지표를 어떻게 재정의하며, TADF 스레드와의 접점/갭은 무엇인가[3]?

\section{WildSci}
\subsection*{(Template note)}
\texttt{WildSci}는 템플릿이 요구하는 메타 섹션이다. 본 리뷰에서는 [1][2]가 공통으로 사용하는 747개 문헌·실험 기반 TADF 코퍼스를 지칭하는 축약 표기로만 사용한다[1][2].

\subsection{정의 및 구성요소}
\subsubsection*{도메인}
유기 TADF 발광체(분자)이며, 구조적 다양성으로 D-A, D-A-D, MR 등 아키텍처를 포함한다[1]. [2]는 동일 분자군을 바이오이미징(NIR), 광촉매, 포토디텍션 등 응용 시나리오로 분기해 선별 맥락을 제시한다[2].
\subsubsection*{태스크}
(가) 계산 화학 기반 속성 산출: $\Delta E_{ST}$, $\lambda_{PL}$, $f$ 등[1].  
(나) 예측(회귀) 모델링: $\Delta E_{ST}$를 ML로 예측하고 SHAP으로 기여도 분석[2].  
(다) 샘플 효율화: 액티브 러닝으로 라벨 획득 전략 비교[2].  
(라) (연결 가능한 외삽 방향의 예시) 속도론 기반 효율/수명 예측: Ir(III)에서 $\Phi$, $\tau$를 경쟁 rate로 계산하며 $k_{nr}(T)$ 포함[3].
\subsubsection*{라벨/레퍼런스}
[1]은 747개 중 실험적으로 측정된 $\Delta E_{ST}$ 312개 및 발광 파장 $\lambda_{PL}$ 213개를 비교 레퍼런스로 사용하고, 수직 근사 기반 HTVS 예측의 실험 대비 MAE가 $\approx 0.17$ eV임을 명시한다[1].

[2]의 핵심 예측 타깃은 $\Delta E_{ST}$이며, 이는 [1]에서 검증한 (GFN2-xTB 기하 + sTDA/sTD-DFT-xTB excited states) 프로토콜을 전제하여 계산된 값을 기반으로 한다고 서술한다[2]. 또한 747 분자에서 ``methods and environments'' 조합으로 2943개 샘플을 구성하고, 5-fold CV 및 held-out test로 회귀 성능(MAE $=0.024$ eV, $R^2=0.960$)을 보고한다[2]. 다만 논문 본문에서 train/test 분할의 \emph{그룹 단위}(분자 기준 vs 샘플 기준)를 명시적으로 고정했다고 확인하기는 어려우므로, 본 리뷰에서는 해당 성능 수치를 \textbf{분할 단위가 보고된 경우에만 비교 가능한 조건부 성능}으로 취급한다[2].

\section{Experiments}
본 섹션은 세 논문에 보고된 계산/모델링 실험의 목적, 가설, 비교군, 변수, 지표를 동일 기준으로 정리하되, 리뷰의 논지(스레드 연결 및 리스크 진단)에 직접 필요한 정보 중심으로 압축한다.

\subsection{실험 목적과 가설}
\subsubsection*{[1] xTB-기반 HTVS 프로토콜 검증}
목적: sTDA-xTB 및 sTD-DFT-xTB가 747개 실험 특성화 TADF 분자에서 HTVS에 충분한 신뢰도로 동작하는지 검증[1].  
가설(서술형): (i) 두 방법은 서로 강한 상관을 가져 상대 랭킹에 유효하고, (ii) 실험과의 절대 오차는 수직 근사 등으로 제한되나 스크리닝 목적에는 수용 가능[1].

\subsubsection*{[2] NTO 디스크립터+ML+AL로 설계 가속}
목적: (i) NTO-기반 CT 디스크립터가 $\Delta E_{ST}$ 예측을 개선하고 물리적 해석가능성을 제공하는지, (ii) 액티브 러닝이 라벨 수요를 절감하는지 평가[2].  
가설: $S_{he}$ 등 CT 디스크립터가 $\Delta E_{ST}$ 예측에 핵심적이며, 불확실성/다양성 기반 쿼리 전략이 무작위보다 샘플 효율적일 것[2].

\subsubsection*{[3] 경쟁 rate 통합 및 온도 의존 소거 포함 효율/수명 예측}
목적: Ir(III) PhOLED에서 경쟁 비활성화 채널(방사/ISC/온도 활성 비복사)을 모두 포함해 $\Phi$와 $\tau$의 온도 의존성을 재현 가능한 프로토콜로 제공[3].  
가설: $k_{nr}(T)$가 상온에서 효율/수명 저하를 지배하며, 이를 명시적으로 포함하면 실험적 온도 곡선을 설명할 수 있음[3].

\subsection{비교군(모델/방법) 및 통제 조건}
[1] 비교군: sTDA-xTB vs sTD-DFT-xTB 상호비교 및 실험값 대비 비교; gas vs toluene(ALPB) 비교[1].  
[2] 비교군: SVR/Gradient Boosting/Random Forest 성능 비교(5-fold CV); AL acquisition function 6종 비교; 랜덤 샘플링 대비 AL 학습곡선 비교[2].  
[3] 비교군: $k_{nr}(T)$ 포함/제거 시 온도 의존 $\tau$ 곡선 변화 및 실험 데이터와의 비교[3].

\subsection{독립/종속변수 및 평가 지표}
[1] 종속: $\Delta E_{ST}$, $\lambda_{PL}$, 상관계수($r$, $\rho$), MAE/RMSE 등; 독립: 방법(sTDA/sTD-DFT), 환경(gas/ALPB toluene), 수직 근사 등[1].  
[2] 종속: $\Delta E_{ST}$ 예측 MAE, $R^2$, SHAP 중요도, AL 학습곡선; 독립: 특징(29개, CT 포함 여부), 모델군(SVR 등), 분할(5-fold CV, held-out), acquisition function 및 초기/배치/반복 설정[2].  
[3] 종속: $\Phi$, $\tau(T)$, rate 구성요소($k_r$, $k_{ISC}$, $k_{nr}(T)$); 독립: 온도, 활성화 장벽 $E_a$, 선택한 계산 프로토콜[3].

\subsection{절차 요약(재현 가능한 수준)}
[1] RDKit로 SMILES$\to$3D 생성 후 CREST+GFN2-xTB로 컨포머 탐색, 최저 에너지 컨포머 최적화, sTDA/sTD-DFT-xTB 단일점 excited-state 계산, ALPB 용매(toluene) 적용, 통계 평가[1].  
[2] 동일 스택에서 NTO 분석으로 CT 디스크립터($\Omega_{CT}$, $\Delta r$, $S_{he}$ 등) 산출, 총 29개 특징으로 ML(특히 SVR RBF) 학습/평가, SHAP으로 중요도 분석, AL에서 RF 트리 분산 기반 불확실성 및 6종 acquisition function 비교(초기 50, 배치 10, 40 iter, 10 seeds)[2].  
[3] $\Phi$와 $\tau$를 경쟁 rate로 정의하고, $k_r$는 Einstein 관계, $k_{ISC}$는 TVCF+TD-DFT/DFT, $k_{nr}(T)$는 3MLCT$\to$3MC 경로 장벽을 포함하는 CVT 기반으로 계산하며, Arrhenius 형태 근사 가능성을 제시[3].

\section{Results and Analysis}
\subsection{핵심 결과(논문별)과 해석}
\subsubsection*{[1] 속도와 내부 일관성은 강점, 실험 절대 정확도는 수직 근사에 의해 제한}
[1]은 747 분자에서 sTDA-xTB와 sTD-DFT-xTB의 $\Delta E_{ST}$ 방법 간 상관이 Pearson $r\approx 0.82$임을 근거로, HTVS에서 상대 랭킹에 유용하다고 주장한다[1]. 동시에 실험 $\Delta E_{ST}$ 312개 대비 MAE $\approx 0.17$ eV를 보고하고, 그 원인을 수직 근사 및 하이브리드 파라미터화(기하 최적화와 excited-state 해밀토니안의 분리)로 설명하며, 정량 예측보다 스크리닝 역할을 명시한다[1]. 또한 전체 747개 계산이 614 CPU hours 수준이며, TD-DFT 대비 99\% 이상 비용 절감을 제시해 속도 기반의 설계 운영 논리를 강화한다[1].

여기서 중요한 구분은, $r\approx 0.82$는 “두 근사 방법의 내부 일관성(내부 랭킹 안정성)”을 보여주지만, 곧바로 “실험 기준 랭킹 성능”을 보장하지는 않는다는 점이다. [1]은 실험과의 절대 오차가 남는다는 사실 자체를 보고하므로, 랭킹 기반 사용 시에도 실험 타당성(특히 임계값 선별의 경계 샘플)에 대한 후속 안전장치가 필요하다는 해석이 필연적으로 따라온다.

\subsubsection*{[2] NTO-CT 디스크립터는 강한 예측 성능을 보고하나, 라벨/분할 정의가 성능 해석의 핵심 조건}
[2]는 29개 디스크립터(에너지, $f$, NTO overlap, CT 지표, proxy 등)를 구성하고, SVR로 $\Delta E_{ST}$ 예측에서 MAE $=0.024$ eV, $R^2=0.960$을 보고한다[2]. SHAP에서 에너지 특징이 $\sim57\%$ (ET1 31\%, ES1 24\%), CT 디스크립터가 $\sim34\%$이며, triplet hole–electron overlap($ST1_{he}$ 표기)가 21\%로 큰 기여를 보인다고 제시하여 “성능+설명가능성”을 동시에 주장한다[2]. 또한 액티브 러닝이 무작위 대비 약 25\% 라벨 절감을 보인다고 보고하고, 초기 50/배치 10/40 iter/10 seeds 및 6종 acquisition function 비교를 제시한다[2].

그러나 [2]의 성능 수치는 해석 조건이 명확히 고정되어야 한다. 첫째, 예측 목표가 실험 $\Delta E_{ST}$인지, 아니면 [1]의 xTB 스택으로 산출된 $\Delta E_{ST}$(계산 라벨)인지에 따라 “의미 있는 과학적 정확도”와 “파이프라인 내부 대리자(surrogate) 정확도”가 구분된다[2]. 둘째, [2]는 747 분자에서 방법/환경 조합으로 2943 샘플로 확장했다고 서술하는데[2], 이때 분할 단위가 \textbf{분자 기준}인지 \textbf{샘플(분자$\times$방법/환경) 기준}인지에 따라 동일 분자의 변형 샘플이 train/test에 동시에 존재하는 누수(leakage) 가능성이 달라진다. [2]는 5-fold CV 및 held-out test를 언급하지만[2], 제공된 본문 근거 범위에서는 fold 구성 단위를 명확히 고정했다고 확인하기 어렵다. 따라서 본 리뷰는 MAE=0.024 eV를 \textbf{분할 단위가 고정될 때에만 비교 가능한 조건부 성능}으로 취급한다.

\subsubsection*{[3] 효율/수명은 경쟁 rate 합성 문제이며, $k_{nr}(T)$가 상온 열화의 지배항이 될 수 있음을 프로토콜로 제시}
[3]은 photoluminescence efficiency $\Phi$와 lifetime $\tau$를 $k_r$, $k_{ISC}$, 그리고 강한 온도 의존 비복사 소거 $k_{nr}(T)$의 경쟁으로 정의한다[3]. 또한 3MLCT$\to$3MC 변환이 흔히 rate-limiting step이며, 복잡한 kinetic scenario를 Arrhenius 형태로 근사한 $k_{nr}(T)$를 제시한다[3]. blue emitter 사례에서 상온(298 K)에서 효율과 수명이 크게 떨어지는 원인을 $k_{nr}(T)$ 활성화로 설명하고, $k_{nr}(T)$ 항을 제거하면 온도 의존성이 소실된다는 비교를 통해 메커니즘 기여를 분리한다[3].

본 스레드에서 [3]의 역할은 “유기 TADF와 동일 메커니즘을 공유한다”는 의미가 아니라, \textbf{목표함수 설계의 형식}을 확장한다는 데 있다. 즉, HTVS/ML이 산출하는 전자구조 지표(예: $\Delta E_{ST}$)를 최종 성능($\Phi$, $\tau$)로 연결하려면 경쟁 채널의 합성 및 온도 의존 항을 포함하는 동역학적 중간층이 필요하다는 점을 정량식으로 보여준다[3].

\subsection{통합 분석: “속도–정확도–목표함수”의 3단 전환}
\subsubsection*{1) 속도 확보: 저비용 전자구조의 운영 논리와 그 한계의 경계 설정}
[1]이 제공하는 핵심은 “대규모로 돌아가는 파이프라인”의 검증이다. TD-DFT 대비 99\% 이상의 비용 절감과 747 분자 규모의 실행 가능성은 설계 탐색의 전제조건을 충족한다[1]. 다만 [1]이 제시한 내부 일관성($r\approx0.82$)은 방법 간 상관이며, 실험 타당성을 대체하지 않는다. 따라서 [1]의 프레이밍(정량 예측이 아닌 스크리닝)을 리뷰에서도 그대로 유지하되, \textbf{실험 기준 의사결정(임계값 선별)}과 \textbf{파이프라인 내부 랭킹}을 구분해 사용해야 한다.

\subsubsection*{2) 정확도/설명가능성 전환: 디스크립터+ML은 ‘무엇을 예측하나’가 성능의 의미를 결정}
[2]는 NTO 기반 $S_{he}$를 포함한 CT 디스크립터로 $\Delta E_{ST}$ 예측 성능을 크게 끌어올렸다고 보고한다[2]. 이는 (i) 저비용 전자구조 출력에서 물리적 정보를 추출해 특징 공간을 정교화하고, (ii) 해당 특징이 실제로 SHAP에서 큰 기여를 갖는다는 정량 근거를 함께 제시했다는 점에서 기술적으로 설득력이 있다[2].

하지만 이 단계에서는 \textbf{라벨의 정합성}이 결정적이다. 만약 목표 $\Delta E_{ST}$가 xTB 스택 산출물이라면, MAE=0.024 eV는 “전자구조 계산의 대리자 모델”로서 강력하나, [1]에서 보고된 실험 대비 MAE 0.17 eV의 갭을 자동으로 해소하지는 않는다[1][2]. 즉, 정확도 전환은 “실험 예측 정확도”라기보다 “파이프라인 내 계산을 더 빠르게 모사하는 정확도”로 이해될 여지가 있으며, 이 해석은 [2]의 데이터 확장(2943)과 분할 방식에 의해 강화되거나 약화된다[2].

\subsubsection*{3) 목표함수 확장: $\Delta E_{ST}$ 중심 설계에서 $\Phi/\tau(T)$로의 전환은 동역학 계층을 요구}
[3]은 효율과 수명이 rate의 함수임을 전면화하고, 특히 $k_{nr}(T)$의 활성화가 상온 열화를 지배할 수 있음을 프로토콜로 제시한다[3]. TADF 스레드에 이 프레임을 접목할 때는, “유기 TADF에서도 동일한 3MC 경로가 지배한다”는 전이를 주장하기보다, \textbf{(a) 경쟁 채널 합성이라는 목표함수 형태}와 \textbf{(b) 온도 의존 항이 최종 성능을 좌우할 수 있다는 구조}를 설계 파이프라인의 상위 요구사항으로 채택하는 것이 적절하다.

\section{Data Quality Analysis}
\subsection{코퍼스 구성과 레퍼런스의 범위}
[1]은 747개의 “실험 특성화 emitters” 코퍼스를 사용하며, 그 중 실험 비교 가능한 레퍼런스로 $\Delta E_{ST}$ 312개 및 $\lambda_{PL}$ 213개를 명시한다[1]. 이는 코퍼스 전체가 동일 품질/동일 레퍼런스를 갖지 않는다는 점(부분 관측)을 의미하며, 모델 검증 시 관측 편향 가능성을 내포한다.

[2]는 동일 747 분자를 전제로 하되, 방법/환경 조합으로 2943 샘플로 확장한다고 서술한다[2]. 이 확장은 데이터량 증가처럼 보이지만, 분자 다양성의 증가가 아니라 \textbf{동일 분자에 대한 조건 변화 샘플링}일 수 있으므로, 평가 설계가 이를 적절히 통제하는지가 핵심이다.

\subsection{라벨 정의와 의미 해석}
[1]에서는 실험 레퍼런스와의 비교를 통해 수직 근사에 기인한 절대 오차를 직접 제시한다(MAE $\approx 0.17$ eV)[1]. 반면 [2]의 $\Delta E_{ST}$ 예측은 “어떤 레퍼런스의 $\Delta E_{ST}$인가”에 따라 의미가 달라진다[2]. 논문은 [1]의 검증된 semi-empirical 프로토콜을 전제로 특징을 구축하므로[2], 리뷰에서는 기본 해석 단위를 “xTB 파이프라인 기반 라벨에 대한 surrogate 가능성”으로 두고, 실험 타당성으로의 전환에는 추가 조건(분자 기준 split, 실험 레퍼런스 재학습/보정)이 필요하다고 정리한다.

\section{Different Data Splits}
\subsection{분할 단위가 만드는 누수(leakage) 리스크}
[2]의 2943 확장 샘플은 분자($M$)와 조건($C$; 방법/환경)의 곱공간으로 이해될 수 있다[2]. 이때 (i) \textbf{샘플 기준 무작위 분할}은 동일 분자 $M$이 서로 다른 조건 $C$로 train/test에 동시에 등장할 수 있어, 분자 수준 일반화라기보다 \textbf{조건 보간}이 성능을 지배할 위험이 있다. (ii) \textbf{분자 기준(그룹) 분할}은 이 누수를 차단하며, “새 분자에 대한 일반화”를 더 엄격히 평가한다.

[2]는 5-fold CV 및 held-out test를 언급하고, 747 분자에서 방법/환경 조합으로 2943 샘플을 구성했음을 명시한다[2]. 그러나 제공된 본문 근거 범위에서는 \emph{fold를 구성하는 분할 단위가 분자(group)인지, 또는 2943 샘플 단위인지}를 명확히 고정하는 문장을 확인하기 어렵다. 따라서 본 리뷰는 MAE=0.024 eV, $R^2=0.960$을 \textbf{(a) 분자 기준 분할이면 “새 분자 일반화” 성능, (b) 샘플 기준 분할이면 “조건 보간 포함” 성능}으로 분리 해석하며, 논문이 분할 단위를 명시하지 않은 경우 해당 수치를 \textbf{비교 가능한 최종 지표가 아니라 조건부(상한 가능) 성능}으로만 인용한다[2].

\section{Statistics of Subdomains}
[1]은 D-A-D 아키텍처의 우수 성능을 통계적으로 확인했다고 보고하며, 도너–억셉터 비틀림 각의 유리 범위를 50$^\circ$–90$^\circ$로 제시한다[1]. [2] 또한 최적 비틀림 각 창을 60$^\circ$–90$^\circ$로 서술하며 CT 디스크립터($S_{he}$)와의 연결을 제시한다[2].

이 두 결과는 “비틀림 각–CT 성격–$\Delta E_{ST}$”라는 공통 축을 제공하지만, 통계가 의미 있으려면 (i) 하위 도메인(아키텍처, 전자구조 유형) 별 표본 수의 불균형, (ii) 실험 레퍼런스가 존재하는 하위집합(312, 213)과의 정합을 함께 점검해야 한다[1]. 원고 수준에서는 논문이 제공하는 범위 내에서 해당 경향의 존재를 요약하고, 도메인별 성능 분해는 후속 과제로 남긴다.

\section{Domain-Specific Performance}
[1]은 전체 코퍼스 수준에서의 상관과 실험 MAE를 제시하지만, 아키텍처별 또는 특정 설계 공간(예: MR vs D-A-D)에서의 오차 구조가 어떻게 달라지는지에 대한 상세 분해는 본문에서 제한적이다[1]. [2]는 응용 시나리오를 제시하나, 시나리오별로 분리된 성능(예: NIR vs blue 등)의 체계적 비교가 리뷰가 참조할 수 있는 형태로 충분히 보고되지는 않는다[2]. 따라서 현재 증거만으로는 “어느 하위 도메인에서 더 잘/못 일반화하는가”를 강하게 주장하기 어렵고, 대신 분할/누수 통제 이후 도메인별 성능을 재보고해야 한다는 요구사항으로 정리한다.

\section{Post-saturation Generalization}
[2]는 액티브 러닝이 무작위 대비 약 25\% 라벨을 절감한다고 보고한다[2]. 다만 이 효율성은 (i) 후보 풀의 분포가 고정된 동일 코퍼스 내에서의 학습 곡선 개선을 의미할 수 있으며, (ii) “포화 이후(더 많은 라벨을 투입한 구간)에서의 외삽/도메인 이동”까지 보장하지는 않는다. [1]과 [2]가 공유하는 747 코퍼스의 경계 밖(새 화학공간) 일반화는 본 논문들만으로는 직접 검증되지 않는다[1][2].

\section{Distribution Illustration}
[1]은 PCA에서 세 성분이 분산의 거의 90\%를 포착한다고 보고하며, 코퍼스가 저차원 구조를 가질 수 있음을 시사한다[1]. 이는 (i) HTVS 단계에서 간단한 구조/전자구조 요약변수로 설계 공간을 압축할 수 있다는 가능성과, 동시에 (ii) ML이 해당 저차원 구조를 이용해 높은 성능을 보일 수 있다는 가능성을 함께 내포한다. 후자의 경우, 분할 누수 또는 유사 분자 군집이 분할을 가로지르는지 여부가 중요해진다.

\section{Experiment Settings}
논문 기반으로 확인 가능한 설정은 다음과 같다. [1]은 RDKit/CREST/GFN2-xTB 기반 컨포머 탐색 후 sTDA-xTB 또는 sTD-DFT-xTB 단일점 excited-state 계산을 수행하고, ALPB(toluene) 용매를 비교한다[1]. [2]는 29개 디스크립터, SVR(RBF) 중심의 모델 비교, SHAP(0.45.0) 분석을 사용하며, 액티브 러닝은 초기 50, 배치 10, 40 반복, 10 seeds, 6종 acquisition function 비교를 명시한다[2]. [3]은 $k_r$ (Einstein), $k_{ISC}$ (TVCF), $k_{nr}(T)$ (CVT, Arrhenius 근사 가능)로 구성된 경쟁 rate 프레임을 사용한다[3].

\section{Related Work}
본 리뷰는 세 논문을 하나의 스레드로 간주하지만, 외부 문헌을 확장해 ‘관련연구 지도’를 작성하지는 않는다(근거 제한). 다만 [1]은 semi-empirical xTB 계열을 HTVS에 적용하는 맥락에서 “screening rather than quantitative prediction”이라는 포지셔닝을 명시하고[1], [2]는 이를 “NTO 기반 디스크립터와 액티브 러닝을 결합한 통합 전략”으로 확장한다[2]. [3]은 효율 계산이 경쟁 채널 때문에 어려웠다는 문제의식을 제시하고, 온도 의존 소거를 포함한 일반 접근이 자동 prescreening에 통합 가능하다고 주장한다[3]. 이 세 문장은 각각의 논문 내부에서 이미 ‘왜 지금 이 접근이 필요한가’를 설명하는 최소한의 관련연구 역할을 수행한다.

\section{Risks \& Gaps}
\subsection{(R1) 수직 근사 기반 절대 오차와 임계값 선별 리스크}
[1]은 실험 $\Delta E_{ST}$ 대비 MAE $\approx 0.17$ eV를 보고하며, 원인을 수직 근사에 귀속한다[1]. 이는 랭킹 중심 스크리닝에는 활용될 수 있으나, $\Delta E_{ST}<0.2$ eV 같은 임계값 기반 의사결정에서는 경계 샘플의 오분류 위험을 직접적으로 증가시킨다(임계값 자체는 [2]가 설계 조건으로 언급)[2].

\subsection{(R2) [1]의 ‘내부 일관성’과 ‘실험 타당성’의 혼동 가능성}
$r\approx 0.82$는 sTDA-xTB와 sTD-DFT-xTB 간 상관이다[1]. 이는 파이프라인 선택의 안정성 근거가 될 수 있지만, 실험 기반 순위 재현성과 동일시되면 과장 해석이 된다. 리뷰는 이를 명시적으로 분리하여 독자 오해를 방지한다.

\subsection{(R3) [2]의 라벨/분할 정합성: 2943 확장 샘플이 만드는 누수 및 과대평가 위험}
[2]의 MAE=0.024 eV가 의미 있으려면, 확장 샘플에서 분자 기준 split을 통해 동일 분자 변형의 train/test 동시 존재를 차단했는지, 그리고 목표 라벨이 무엇인지가 함께 고정되어야 한다[2]. 이 조건이 충족되지 않으면 해당 성능은 “동일 분자 조건 변화에 대한 보간” 성격이 강해질 수 있다.

\subsection{(R4) 고체상/호스트/형태학 효과의 부재}
[1]은 gas vs ALPB toluene로 환경을 비교하지만, OLED 실제 조건(고체상, 호스트 매트릭스, 농도/응집, 결정/비정질 형태)의 영향은 본 스레드의 전자구조/ML 파이프라인에 직접 통합되지 않는다[1][2]. 이는 실험으로의 전이에서 구조적 갭으로 남는다.

\subsection{(R5) 목표함수 불일치: $\Delta E_{ST}$ 중심 설계와 $\Phi/\tau(T)$ 중심 성능의 간극}
[2]는 $\Delta E_{ST}$와 $k_{RISC}$ 조건을 병기하며 설계의 동역학 조건을 언급하지만[2], 실제로는 $\Delta E_{ST}$ 예측에 집중한다. [3]은 효율/수명이 경쟁 rate의 합성이라는 점과 $k_{nr}(T)$의 지배 가능성을 정식화한다[3]. 이 둘 사이에는 “전자구조 지표를 동역학·온도 성능으로 맵핑하는 계층”이 비어 있다.

\section{Critics}
편집자 관점의 핵심 비평은 다음과 같다. (i) [2]의 매우 낮은 MAE는 분할 단위/라벨 정의 없이는 과학적 성능 주장으로 단정하기 어렵다[2]. (ii) [1]의 랭킹 적합성은 내부 상관으로 지지되지만, 실험 기준 의사결정으로 확대 적용할 때는 수직 근사 오차가 선별 성능을 제한할 수 있다[1]. (iii) [3]의 속도론 프레임은 목표함수 확장에 유용하나, 물질군 차이(Ir(III) vs 유기 TADF)를 넘어 메커니즘을 직접 전이하는 방식의 일반화는 본 세 논문의 증거만으로는 정당화되지 않는다[3].

\section{Limitations}
\subsection{내적 타당도(실험 설계/비교의 공정성)}
\subsubsection*{수직 근사 및 하이브리드 해밀토니안의 구조적 편향}
[1]의 핵심 한계는 S0 기하에서의 수직 근사를 HTVS 전제조건으로 두는 점이며, 이로 인한 실험 대비 MAE $\sim0.17$ eV가 관측된다[1]. 이 한계는 랭킹 목적에서는 일부 완화되나, 임계값 기반 선별에서는 경계 샘플의 false positive/negative를 유발할 수 있다. [1]은 정량 예측에는 더 높은 수준의 방법이 필요함을 전제한다[1].

\subsubsection*{[2]의 “확장 샘플(2943)”과 분할 단위의 누수 위험(완결)}
[2]는 747 분자에서 ``methods and environments'' 조합으로 2943 샘플로 확장했다고 서술한다[2]. 이 확장은 데이터 수를 늘리지만, 동일 분자에 대한 조건 변형 샘플이 다수 포함될 수 있다. 따라서 샘플 단위 무작위 분할은 동일 분자 군이 train과 test에 동시에 존재하는 누수 위험을 내포한다. [2]는 5-fold CV 및 held-out test를 보고하지만[2], 분자 기준 분할 여부가 명시되지 않으면 MAE=0.024 eV가 “새 분자 일반화”가 아니라 “동일 분자 조건 보간”을 반영할 수 있다. 리뷰 관점에서는 (i) 분자 기준 split을 기본으로 보고, (ii) 샘플 기준 split 성능은 별도로 보고하며, (iii) 확장 샘플을 사용할 경우 분자 ID를 그룹으로 묶는 group CV가 필요하다는 점을 제한으로 정리한다.

\subsection{외적 타당도(도메인 이동/실험 조건)}
[1][2]의 코퍼스는 문헌 기반이지만, 실제 OLED 소자 조건(호스트, 도핑, 형태학, 계면)으로의 전이는 별도 검증이 필요하다[1][2]. [3]은 자동 prescreening 통합 가능성을 주장하지만[3], TADF 유기체계에 동일한 $k_{nr}(T)$ 기작을 적용하는 것은 본 증거만으로는 단정할 수 없다.

\section{Key Findings}
\subsection*{F1} 747 분자 규모에서 xTB 기반 sTDA/sTD-DFT는 TD-DFT 대비 99\%+ 비용 절감과 방법 간 높은 내부 상관을 제공하여 HTVS의 실행 가능성을 입증한다[1].
\subsection*{F2} 실험 $\Delta E_{ST}$에 대한 절대 오차(MAE $\approx 0.17$ eV)는 수직 근사에 의해 구조적으로 제한되며, 랭킹 중심 사용과 임계값 선별은 구분되어야 한다[1].
\subsection*{F3} NTO 기반 CT 디스크립터($S_{he}$ 등)는 $\Delta E_{ST}$ 예측에서 높은 성능과 해석가능성(SHAP 기여)을 동시에 보고한다[2].
\subsection*{F4} [2]의 성능 주장은 라벨 정의(실험 vs 계산)와 분할 단위(분자 vs 샘플)가 고정되어야 비교 가능하며, 확장 샘플(2943)은 누수 위험을 내포한다[2].
\subsection*{F5} 효율/수명은 경쟁 rate 합성 문제이며, 온도 의존 $k_{nr}(T)$가 상온 성능을 지배할 수 있다는 동역학적 목표함수 프레임이 제공된다[3].

\section{Trends \& Implications}
본 스레드가 제시하는 추세는 (i) “대규모로 돌아가는 저비용 전자구조”의 실증[1], (ii) 그 출력에서 물리적 디스크립터를 추출해 ML과 액티브 러닝으로 탐색 효율을 극대화하는 전환[2], (iii) 최종 의사결정 변수를 에너지 갭에서 효율/수명·온도 민감도로 확장하는 요구[3]로 요약된다. 기술 리드 관점에서의 함의는, HTVS 시스템은 더 이상 단일 지표의 빠른 계산에 머물 수 없고, (a) 라벨/분할의 엄격한 평가 인프라, (b) 고체상/소자 조건을 반영하는 도메인 적응, (c) 동역학적 목표함수 계층의 통합을 포함하는 “전체 파이프라인 엔지니어링” 단계로 진입해야 한다는 것이다.

\section{Prompts}
본 섹션은 템플릿 요구를 충족하기 위한 기록으로, 본 리뷰가 사용한 분석 프롬프트의 요지를 요약한다. (i) 세 논문을 HTVS$\to$디스크립터/ML/AL$\to$동역학 목표함수 확장으로 연결하라. (ii) 논문 내 정량 수치(코퍼스 규모, MAE, $R^2$, 상관, AL 설정/효과, kinetic 방정식)를 근거로 삼고, 외부 추측을 제한하라. (iii) 리스크는 라벨 정의, 분할 누수, 수직 근사, 고체상/호스트 부재, 목표함수 불일치로 구조화하라.

\section{Conclusion}
세 논문은 “속도 확보(저비용 전자구조의 대규모 검증)”[1]에서 “정확도/설명가능성(물리 디스크립터+ML/AL)”[2]로, 그리고 “목표함수 확장(효율/수명·온도 동역학)”[3]으로 이어지는 가속 설계 스레드를 형성한다. 다만 이 스레드가 실제 설계 의사결정으로 작동하기 위해서는, 수직 근사로 인한 절대 오차와 임계값 선별 리스크[1], 확장 샘플과 분할 단위가 만드는 누수 가능성[2], 그리고 $\Delta E_{ST}$ 중심 스크리닝과 $\Phi/\tau(T)$ 중심 성능의 구조적 간극[3]을 해소하는 통합 연구가 필요하다.

\subsection{Actionable research directions (3--5) \& next-step questions}
\subsubsection*{D1. 분자 기준(group) 분할을 표준으로 하는 재평가 리포팅}
질문: [2]의 2943 확장 샘플에서 분자 기준 group CV/held-out을 적용하면 MAE=0.024 eV, $R^2=0.960$이 얼마나 유지되는가[2]?

\subsubsection*{D2. 실험 레퍼런스(312) 중심의 2단계 보정: 랭킹$\to$정량 재평가}
질문: [1]이 제시한 수직 근사 오차(MAE $\approx0.17$ eV)를 고려할 때, HTVS 1단 랭킹 후 경계 구간만 고정밀 재평가하는 2단 파이프라인이 false positive/negative를 얼마나 줄이는가[1]?

\subsubsection*{D3. 라벨 정합성 명시: ‘계산 라벨 surrogate’와 ‘실험 예측’의 목표 분리}
질문: [2]의 학습 목표를 (i) xTB 산출 $\Delta E_{ST}$, (ii) 실험 $\Delta E_{ST}$로 분리하여 각각의 성능과 실패 모드를 보고하면, 어떤 디스크립터($S_{he}$ 포함)가 어느 목표에서 일관되게 유효한가[1][2]?

\subsubsection*{D4. 목표함수 확장 실험: $\Delta E_{ST}$ 기반 선별을 경쟁 rate 계층으로 연결}
질문: [3]의 경쟁 rate 프레임(특히 온도 의존 $k_{nr}(T)$)을 참고하여, TADF 스크리닝에서도 “상온 열화에 민감한 비복사 경로”를 목표함수에 포함시키려면 어떤 중간층(rate 모델 또는 proxy)이 최소 구성으로 필요하며, 그 proxy는 [1][2]의 저비용 전자구조 출력에서 추출 가능한가[1][3]?

\subsubsection*{D5. 액티브 러닝의 불확실성 추정 검증: 효율성 주장의 조건부화}
질문: [2]에서 사용한 RF 트리 분산 기반 불확실성이 실제 일반화 오차(특히 분자 기준 split)와 얼마나 정렬되는가, 그리고 acquisition function 간 우열이 분할 설정에 따라 유지되는가[2]?

\section{Appendix}
\subsection{Citation mapping}
[1] arXiv:2511.00922v1.  
[2] arXiv:2512.06029v1.  
[3] arXiv:1901.01201v1.
